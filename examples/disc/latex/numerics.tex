%\documentclass[12pt,a4paper]{amsart}
%!TEX TS-options = --shell-escape
\pdfobjcompresslevel=0% fix PDF inclusions
\documentclass[%draft,
a4paper,11pt,DIV=11,% decrease borders by 2 (std 10 for 11pt)
abstract=on% return to old style centered abstract
]{scrartcl}
%
\usepackage{algorithm, booktabs}
\usepackage{mathtools}
\mathtoolsset{showonlyrefs}
\usepackage{amsfonts}
\usepackage{amsmath}
\usepackage{amssymb}
\usepackage{faktor}
\usepackage{amscd}
\usepackage{array}
%\usepackage{refcheck}\norefnames
\usepackage{amsthm}
\usepackage{subfigure}
\usepackage{mathdots}
\usepackage{mathrsfs}
\usepackage[noend]{algpseudocode}
\usepackage{graphicx}
\usepackage{color}
\usepackage{pgf}
\usepackage{scrpage2}
\usepackage{multirow}
\usepackage{listings}
\lstset{language=matlab,showstringspaces=false,basicstyle={\ttfamily}}
\usepackage{tikz}
\usepackage{graphicx}
\usepackage[left=2.5cm, right=2.5cm,top=2.5cm,bottom=2.5cm]{geometry}
\usepackage{scalerel}

\newcommand{\weakly}{\rightharpoonup}
\newcommand{\N}{\ensuremath{\mathbb{N}}}
\newcommand{\T}{\ensuremath{\mathbb{T}}}
\renewcommand{\S}{\ensuremath{\mathbb{S}}}
\newcommand{\B}{\ensuremath{\mathbb{B}}}
\newcommand{\NZ}{\ensuremath{\mathbb{N}_{0}}}
\newcommand{\Z}{\ensuremath{\mathbb{Z}}}
\newcommand{\Q}{\ensuremath{\mathbb{Q}}}
\newcommand{\R}{\ensuremath{\mathbb{R}}}
\newcommand{\C}{\ensuremath{\mathbb{C}}}
\newcommand{\nchoosek}[2]{\left(\begin{array}{c}#1\\#2\end{array}\right)}
\newcommand{\ii}{\textnormal{i}}
\newcommand{\X}{\mathbb{X}}
\newcommand{\Y}{\mathbb{Y}}
\newcommand{\e}{\textnormal{e}}
\newcommand{\supp}{\textnormal{supp}}
\newcommand{\eip}[1]{\textnormal{e}^{2\pi\ii{#1}}}
\newcommand{\eim}[1]{\textnormal{e}^{-2\pi\ii{#1}}}
\newcommand{\norm}[1]{\left\Vert #1\right\Vert}
\def\invisible#1{\textcolor{white}{#1}}
\newcommand{\ceil}[1]{\encl{\lceil}{#1}{\rceil}}
\newcommand{\floor}[1]{\encl{\lfloor}{#1}{\rfloor}}
\newcommand{\zb}[1]{\ensuremath{\boldsymbol{#1}}}
\newcommand{\pmat}[1]{\begin{pmatrix} #1 \end{pmatrix}}
\newcommand{\set}[1]{\left\{ #1 \right\}}
\newcommand{\abs}[1]{\left| #1 \right |}
\newcommand{\dx}{\,\mathrm{d}}
\newcommand{\tT}{\mathrm{T}}
\newcommand{\uT}{\scriptscriptstyle{T}}
\newcommand{\uuT}{{\scaleto{T}{3pt}}}
\newcommand{\TSP}{\text{TSP}}

\def\3{\ss}
\def\ep{\varepsilon}
\def\la{\lambda}
\def\La{{\mathbf\lambda}}
\def\Ga{\Gamma}
\def\fl#1{\lfloor#1\rfloor}
\def\cl#1{\lceil#1\rceil}

\newmuskip\pFqmuskip

\newcommand*\pFq[6][8]{
  \begingroup % only local assignments
  \pFqmuskip=#1mu\relax
  % make the comma math active
 % \mathcode`\,=\string"8000
  % and define it to be \pFqcomma
  \begingroup\lccode`\~=`\,
  \lowercase{\endgroup\let~}\pFqcomma
  % typeset the formula
  {}_{#2}F_{#3}{\left(\genfrac..{0pt}{}{#4}{#5};#6\right)}%
  \endgroup
}
\newcommand*\pRegFq[6][8]{
  \begingroup % only local assignments
  \pFqmuskip=#1mu\relax
  % make the comma math active
 % \mathcode`\,=\string"8000
  % and define it to be \pFqcomma
  \begingroup\lccode`\~=`\,
  \lowercase{\endgroup\let~}\pFqcomma
  % typeset the formula
  {}_{#2}\tilde{F}_{#3}{\left(\genfrac..{0pt}{}{#4}{#5};#6\right)}%
  \endgroup
}


\newcommand{\pFqcomma}{\mskip\pFqmuskip}


\newcommand*{\bigtimes}{\mathop{\raisebox{-.5ex}{\hbox{\Large{$\times$}}}}}

\DeclareMathOperator{\Lip}{Lip}
\DeclareMathOperator*{\AC}{AC^p}
\DeclareMathOperator*{\Speed}{Speed}
\DeclareMathOperator*{\Length}{Length}
\DeclareMathOperator*{\argmax}{arg\,max}
\DeclareMathOperator*{\diag}{diag}
\DeclareMathOperator*{\trace}{trace}
\DeclareMathOperator*{\diam}{diam}
\DeclareMathOperator*{\spann}{span}
\DeclareMathOperator{\dist}{dist}
\DeclareMathOperator{\F}{F}
\DeclareMathOperator*{\maxdeg}{maxdeg}
\DeclareMathOperator*{\sep}{sep}
\DeclareMathOperator*{\lt}{lt}
\DeclareMathOperator*{\Var}{Var}
\DeclareMathOperator*{\Beta}{Beta}
\DeclareMathOperator{\G}{\mathcal{G}}
\DeclareMathOperator{\geo}{geo}
\DeclareMathOperator{\ch}{ch}
\DeclareMathOperator*{\per}{per}
\DeclareMathOperator*{\id}{id}
\DeclareMathOperator*{\vol}{vol}
\DeclareMathOperator{\Pol}{Pol}
\DeclareMathOperator*{\Real}{Re}
\DeclareMathOperator*{\Imag}{Im}
\DeclareMathOperator*{\disc}{disc}
\DeclareMathOperator{\OOO}{O}
\DeclareMathOperator*{\SO}{SO}
\DeclareMathOperator*{\argmin}{argmin}
\DeclareMathOperator*{\Mapprox}{\mathcal{P}_{\text{res}}(\mathcal M)}
\DeclareMathOperator{\Space}{\mathcal{M}}


\newtheorem{thm}{Theorem}[section]
\newtheorem{lemma}[thm]{Lemma}
\newtheorem{remark}[thm]{Remark}
\newtheorem{definition}[thm]{Definition}
\newtheorem{example}[thm]{Example}
\newtheorem{corollary}[thm]{Corollary}
\newtheorem{proposition}[thm]{Proposition}

%---------------------------------------------------------------------------
\begin{document}
\title{Curve Based Approximation of Measures on Manifolds by Discrepancy Minimization}

\author{
Martin Ehler\footnotemark[1],
\and
Manuel Gr\"af\footnotemark[2],
\and
Sebastian Neumayer\footnotemark[3]
\and
and Gabriele Steidl\footnotemark[3]
}
\maketitle

\date{\today}


\footnotetext[3]{Department of Mathematics,
	TU Kaiserslautern,
	Paul-Ehrlich-Str.~31, D-67663 Kaiserslautern, Germany,
	\{neumayer,steidl\}@mathematik.uni-kl.de
	}
\footnotetext[1]{University of Vienna, Department of Mathematics, Vienna, Austria,
	\{martin.ehler\}@univie.ac.at
	}
\footnotetext[2]{Austrian Academy of Sciences, Acoustics Research Institute, Vienna, Austria,
	\{mgraef\}@kfs.oeaw.ac.at
	}

\begin{abstract}
	\noindent\small
	

\end{abstract}

\section{Main problem}

We like to solve
\begin{equation}
  \label{eq:min_PL}
  \inf_{\nu} \mathcal D(\nu,\mu) \qquad \text{subject to} \qquad \nu \in \mathcal P_{L}^{\mathrm{curve}}(\mathbb X). 
\end{equation}


\section{Numerical Approach}

For fixed $L>0$ and probability measure $\mu$ on $\mathbb X$ we aim to solve the minimization problem
\begin{equation}
  \label{eq:constant_speed_min}
  \min_{x_{1},\dots,x_{N}} \mathscr{D}^{2} \Big(\mu, \frac{1}{N} \sum_{k=1}^{N} \delta_{x_{k}}\Big) \qquad \mathrm{s.t.} \qquad    \mathrm{dist}(x_{1},x_{2}) =  \mathrm{dist}(x_{2},x_{3}) = \dots = \mathrm{dist}(x_{N},x_{1}) =  \frac LN.
\end{equation}
Letting $N$ tend to infinity we eventually approach a closed curve $\gamma$, $\gamma(\tfrac kN) \approx x_{k}$ of constant speed $L$ such that the push-forward $\nu = \gamma_{*}(\omega)$, of the Lebesgue measure $\omega$, approaches $\mu$.  

We shall solve the optimization problem approximately by applying the nonlinear conjugate gradient method, cf. {\color{blue} Section ...}, to the following unconstrained formulation
\begin{equation}
  \label{eq:constant_speed_min_relaxed}
  \min_{x_{1},\dots,x_{N}} \mathscr{D}^{2} \Big(\mu, \frac{1}{N} \sum_{k=1}^{N} \delta_{x_{k}}\Big) +   \frac{\alpha}{N} \sum_{k=1}^{N} \big(N \mathrm{dist}(x_{k-1},x_{k}) - L \big)^{2}
\end{equation}
where $\alpha > 0$ is a penalty parameter. Note, for $\alpha \to \infty$ there is a subsequence of minimizers of \eqref{eq:constant_speed_min_relaxed} which will converge to minimizers of \eqref{eq:constant_speed_min}.

Recall that for given kernels $K:\mathbb X\times \mathbb X \to \mathbb R$ the squared discrepancy can be evaluated by
\[
  \mathscr D^{2}\Big(\mu, \frac 1N \sum_{k=1}^{N} \delta_{x_{k}}\Big) = E_{\mu}(x_{1},\dots,x_{N})
\]
where
\begin{equation}
  \label{eq:E_mu}
  E_{\mu}(x_{1},\dots,x_{N}) :=
    \frac{1}{N^{2}}\sum_{i,j=1}^{N} K(x_{i},x_{j}) - 2\sum_{i=1}^{N} \int_{\mathbb X} K(x_{i},x) \mathrm d\mu(x) + \int_{\mathbb X}\int_{\mathbb X} K(x,y) \mathrm d\mu(x) \mathrm d\mu(y).
  \end{equation}
Using the eigenfunction expansion
\[
 K(x,y) = \sum_{l=0}^{\infty} \lambda_{l} \varphi_{l}(x) \overline{\varphi_{l}(y)} 
\]
we have the alternative evaluation formula
\begin{equation}
  \label{eq:E_mu_Fourier}
  E_{\mu}(x_{1},\dots,x_{N}) = \sum_{l=0}^{\infty} \lambda_{l} \Big| \hat\mu_{l}
  - \frac 1N \sum_{i=1}^{N} \varphi_{l}(x_{i}) \Big|^{2}, \quad\quad \mu_{l} := \int_{\mathbb X} \varphi_{l}(x) \mathrm d\mu(x).
\end{equation}
Both formulas have it pros and cons:
\begin{itemize}
\item The formula \eqref{eq:E_mu} allows for exact evaluation only if the expression
\[
  K_{\mu}(x) := \int_{\mathbb X} K(x,y) \mathrm d\mu(y), \qquad  c_{\mu} := \int_{\mathbb X} K_{\mu}(x) \mathrm d\mu(x),
\]
can be written in closed form. In that case the complexity scales quadratically in the number $N$ of points.

\item The formula \eqref{eq:E_mu_Fourier} allows for exact evaluation only if the kernel has a finite expansion. In that case the complexity scales linearly in the number $N$ of points. 
\end{itemize}
Our approach is to approximate the kernel $K$ by a truncation
\[
  K_{t}(x,y) := \sum_{l=0}^{t} \lambda_{l} \varphi_{l}(x) \overline{\varphi_{l}(y)}
\]
with $t \sim N$ and evaluate formula \eqref{eq:E_mu_Fourier} by fast summation methods, so that the overall complexity is almost linearly in $t$ and $N$. 

{\color{blue} To support the reasonablenes of this approach we may compare the minimizers of $K$ with $K_{t}$ for the uniform measure $\mu$.}
\subsection{Choice of Parameters}

In order to find reasonable (local) solutions of \eqref{eq:min_PL} we shall carefully adjust the parameters in problem \eqref{eq:constant_speed_min_relaxed}, namely $N$, $\alpha$, and the polynomial degree $t$.

\textbf{Number of Points.}
From the asymptotic of the path lengths of the traveling sales man problem (TSP) {\color{blue}(see aslo asymptotic for atomic/empirical measures)}
we know that we should choose
\[
  N \sim L(\gamma)^{\frac{d}{d-1}},
\]
where
\[
  L(\gamma) = \sum_{k=1}^{N} \mathrm{dist}(x_{k-1},x_{k})
\]
is the length of the resulting constant speed curve $\gamma$ going through the points, i.e.,
\[
 \gamma(t_{k}) := x_{k}, \qquad t_{k} = \sum_{i=1}^{k}  \mathrm{dist}(x_{i-1},x_{i})/ L(\gamma).
\]

In Figure~\ref{fig:Nexperiment} {(\color{blue} Example environment..?)} we show some results for the two-dimensional torus. 

\textbf{Penalty Parameter.}
Intuitively, the minimizers of \eqref{eq:constant_speed_min_relaxed} will treat both terms almost equally, i.e., for $N\to \infty$ both terms are of the same order. Hence, our heuristic is to chose the parameter $\alpha$, such that the objective function and the penalty term are of the same order for minimizers of \eqref{eq:constant_speed_min_relaxed} i.e.,
\begin{equation}
  \label{eq:penalty_behavior}
  \frac{\alpha}{N} \sum_{k=1}^{N} \big(N \mathrm{dist}(x_{k-1},x_{k}) - L \big)^{2} \sim N^{-\frac{2s}{d}} \sim
  \min_{x_{1},\dots,x_{N}} \mathscr{D}^{2} \Big(\mu, \frac{1}{N} \sum_{k=1}^{N} \delta_{x_{k}}\Big) 
\end{equation}
Assuming that for the length $L(\gamma_{N}) = \sum_{k=1}^{N}\mathrm{dist}(x_{k-1},x_{k})$ of a minimizer $\gamma$ we have $L(\gamma) \sim L$ and that $\mathrm{dist}(x_{k-1},x_{k}) \approx \frac1N L(\gamma)$, then the value of the penalty term behaves like
\[
  \frac{\alpha}{N} \sum_{k=1}^{N} \big(N \mathrm{dist}(x_{k-1},x_{k}) - L \big)^{2} \sim  \alpha L^{2} \sim \alpha N^{\frac{2d}{d-1}}
\]
So that a reasonable choice is
\[
  \alpha \sim L^{-\frac{2s}{d-1}-2} \sim N^{\frac{-2s-2d+2}{d}}.
\]

In Figure~\ref{fig:L_alpha_experiment} {(\color{blue}Example environment..?)} we show some results for the two-dimensional torus. 

\textbf{Polynomial degree.} We believe it is reasonable to take $t \sim N^{\frac1d} \sim L^{\frac{1}{d-1}}$. Note, in that case the computational cost is almost linearly in $N$, for $\mathbb X = \mathbb T^{d}, \mathbb S^{2}, \mathrm{SO(3)}, \mathcal G_{2,4}$. In Figure~\ref{fig:t_experiment} {(\color{blue}Example environment..?)} we show some results for the two-dimensional torus. 



\subsection{Basic Observations}
For large $L$ we cannot hope to find a solution/global minimizer to problem \eqref{eq:constant_speed_min_relaxed}, since the number of local minimizers appears to increase exponentially {\color{blue}[ ref???]}.

The intention of this section is to provide a case study of the influence/interplay of certain parameters to (local) solutions of
problem \eqref{eq:constant_speed_min_relaxed}, namely
\begin{itemize}
\item the parameters $L$, $N$, $\alpha$
\item initial point/curve
\item choice of kernel (truncation parameter $t$)
\item choice of measure $\mu$ (atomic, singular, regular, disconnected support,...) 
\end{itemize}

In what follows we apply the conjugate gradient method to problem \eqref{eq:constant_speed_min_relaxed} for the torus $\mathbb T^{2}$.

\textbf{Influence of the number of points.} We like to show the dependence on the parameter $N$. In this experiment we increase for fixed $L$ and $\alpha$ the number $N$ of points by subdividing the line segments.  We let $\mu$ be the Lebesgue measure, $L=4$, $\alpha=0.1\cdot 2^{-\frac52 i}$, $i=1,\dots,5$, and consider the kernel $K_{t}$, $t=32$. We start with a local minimizer for $N=10$. Then, we successively increase $N$ by inserting the mid points of the line segments connecting $x_{i}$ and $x_{i+1}$ and perform a local minimization. The results are depicted in Figure~\ref{fig:Nexperiment}.

Note, the choice $\alpha =0.1 (N/5)^{-\frac 52}$ on the diagonal in  Figure~\ref{fig:Nexperiment} {\color{blue}(see section: choice of alpha)}. This shows that the number of points leads to reasonable approximations, where the main features of the curves are preserved.

\begin{figure}
  \begin{tabular}{ccccc}
$N=10$  &  $N=20$   & $N=40$  & $N=80$  & $N=160$ \\
  \includegraphics[width=0.18\textwidth]{Images/N10_L4_a1000} & 
  \includegraphics[width=0.18\textwidth]{Images/N20_L4_a1000} & 
  \includegraphics[width=0.18\textwidth]{Images/N40_L4_a1000} &
  \includegraphics[width=0.18\textwidth]{Images/N80_L4_a1000} &
  \includegraphics[width=0.18\textwidth]{Images/N160_L4_a1000}\\
$L(\gamma)\approx 4.0000$ &
$L(\gamma)\approx 4.0000$ &
$L(\gamma)\approx 4.0000$ &
$L(\gamma)\approx 4.0000$ &
$L(\gamma)\approx 4.0000$\\ 
  \includegraphics[width=0.18\textwidth]{Images/N10_L4_i1} & 
  \includegraphics[width=0.18\textwidth]{Images/N20_L4_i1} & 
  \includegraphics[width=0.18\textwidth]{Images/N40_L4_i1} &
  \includegraphics[width=0.18\textwidth]{Images/N80_L4_i1} &
  \includegraphics[width=0.18\textwidth]{Images/N160_L4_i1}\\
$L(\gamma)\approx 3.72$ &
$L(\gamma)\approx 4.20$ &
$L(\gamma)\approx 4.43$ &
$L(\gamma)\approx 4.49$ &
$L(\gamma)\approx 4.50$\\ 
  \includegraphics[width=0.18\textwidth]{Images/N10_L4_i2} & 
  \includegraphics[width=0.18\textwidth]{Images/N20_L4_i2} & 
  \includegraphics[width=0.18\textwidth]{Images/N40_L4_i2} &
  \includegraphics[width=0.18\textwidth]{Images/N80_L4_i2} &
  \includegraphics[width=0.18\textwidth]{Images/N160_L4_i2}\\
$L(\gamma)\approx 3.68$ &
$L(\gamma)\approx 4.47$ &
$L(\gamma)\approx 5.16$ &
$L(\gamma)\approx 5.38$ &
$L(\gamma)\approx 5.44$\\ 
  \includegraphics[width=0.18\textwidth]{Images/N10_L4_i3} & 
  \includegraphics[width=0.18\textwidth]{Images/N20_L4_i3} & 
  \includegraphics[width=0.18\textwidth]{Images/N40_L4_i3} &
  \includegraphics[width=0.18\textwidth]{Images/N80_L4_i3} &
  \includegraphics[width=0.18\textwidth]{Images/N160_L4_i3}\\
$L(\gamma)\approx 3.68$ &
$L(\gamma)\approx 4.66$ &
$L(\gamma)\approx 5.91$ &
$L(\gamma)\approx 6.64$ &
$L(\gamma)\approx 6.87$\\ 
  \includegraphics[width=0.18\textwidth]{Images/N10_L4_i4} & 
  \includegraphics[width=0.18\textwidth]{Images/N20_L4_i4} & 
  \includegraphics[width=0.18\textwidth]{Images/N40_L4_i4} &
  \includegraphics[width=0.18\textwidth]{Images/N80_L4_i4} &
  \includegraphics[width=0.18\textwidth]{Images/N160_L4_i4}\\
$L(\gamma)\approx 3.69$ &
$L(\gamma)\approx 4.73$ &
$L(\gamma)\approx 6.45$ &
$L(\gamma)\approx 8.15$ &
$L(\gamma)\approx 9.03$\\ 
  \includegraphics[width=0.18\textwidth]{Images/N10_L4_i5} & 
  \includegraphics[width=0.18\textwidth]{Images/N20_L4_i5} & 
  \includegraphics[width=0.18\textwidth]{Images/N40_L4_i5} &
  \includegraphics[width=0.18\textwidth]{Images/N80_L4_i5} &
  \includegraphics[width=0.18\textwidth]{Images/N160_L4_i5}\\
$L(\gamma)\approx 3.69$ &
$L(\gamma)\approx 4.74$ &
$L(\gamma)\approx 6.73$ &
$L(\gamma)\approx 9.02$ &
$L(\gamma)\approx 11.25$\\ 
  \end{tabular}
  \caption{Local minimizers  of \eqref{eq:constant_speed_min_relaxed} for $\mu$ the Lebesgue measure, $L=4$ with kernel $K_{t}$, $t=64$ and increasing $N$ (column-wise) and decreasing $\alpha$ (row-wise). The first row shows the local minimizers for  $\alpha=1000$ and corresponds to curves of constant speed and length $L=4$. In the following rows local minimizers are depicted for $\alpha = 0.1\cdot 2^{-\frac52 i}$,$i=1,\dots,5$. Note, the increase in the length of the curves for decreasing $\alpha$ or increasing $N$, until stagnation for sufficient small $\alpha$ or big $N$. For all minimizer the distance between consecutive points is around $L(\gamma)/N$.}
  \label{fig:Nexperiment}
\end{figure}

\textbf{Influence of the penalty parameter.} In this experiment we like to give an intuition of the parameter $\alpha$. There we choose $\alpha := c L^{-5}$ and observe that indeed the (local) minimizers have length almost proportional to $L$. In the experiment we chose $N$ sufficiently large, see Figure~\ref{fig:Nexperiment}. We note that the distance of consecutive points do vary more for smaller constants $c$, i.e., for higher ratios $L(\gamma)/L$. In other words the measure $\omega$ along the constant speed curve differ more from the Lebesgue measure. This corresponds to the fact that $\alpha$ needs to tend to infinity in order to solve problem \eqref{eq:constant_speed_min}.
\begin{figure}
  \begin{tabular}{cccc}
    % $L=1$  &
               $L=2$ &  $L=4$  & $L=8$ & $L=16$\\
%  \includegraphics[width=0.23\textwidth]{Images/Ramp_L1_i1} & 
  \includegraphics[width=0.23\textwidth]{Images/Ramp_L2_i1} & 
  \includegraphics[width=0.23\textwidth]{Images/Ramp_L4_i1} & 
  \includegraphics[width=0.23\textwidth]{Images/Ramp_L8_i1} &
  \includegraphics[width=0.23\textwidth]{Images/Ramp_L16_i1} \\
%$L(\gamma)\approx 3.92$ &
$L(\gamma)\approx 8.01$ &
$L(\gamma)\approx 16.23$ &
$L(\gamma)\approx 32.62$ &
$L(\gamma)\approx 65.90$ \\
%  \includegraphics[width=0.23\textwidth]{Images/Ramp_L1_i2} & 
  \includegraphics[width=0.23\textwidth]{Images/Ramp_L2_i2} & 
  \includegraphics[width=0.23\textwidth]{Images/Ramp_L4_i2} & 
  \includegraphics[width=0.23\textwidth]{Images/Ramp_L8_i2} &
  \includegraphics[width=0.23\textwidth]{Images/Ramp_L16_i2} \\
%$L(\gamma)\approx 1.90$ &
$L(\gamma)\approx 3.94$ &
$L(\gamma)\approx 8.05$ &
$L(\gamma)\approx 16.31$ &
$L(\gamma)\approx 32.72$ \\
%  \includegraphics[width=0.23\textwidth]{Images/Ramp_L1_i3} & 
  \includegraphics[width=0.23\textwidth]{Images/Ramp_L2_i3} & 
  \includegraphics[width=0.23\textwidth]{Images/Ramp_L4_i3} & 
  \includegraphics[width=0.23\textwidth]{Images/Ramp_L8_i3} &
  \includegraphics[width=0.23\textwidth]{Images/Ramp_L16_i3} \\
%$L(\gamma)\approx 1.01$ &
$L(\gamma)\approx 2.03$ &
$L(\gamma)\approx 4.08$ &
$L(\gamma)\approx 8.21$ &
$L(\gamma)\approx 16.46$ \\
  \end{tabular}
   \caption{Local minimizers  of \eqref{eq:constant_speed_min_relaxed} for $\mu$ the measure given by the density $\frac{2\pi-y}{4\pi^{3}} \mathrm dx \mathrm dy$, with kernel $K_{t}$, $t=128$ and several choices of $L$ and $\alpha$. $L$ is increasing column-wise and row-wise we set $\alpha = c L^{-5}$. The constants $c=0.003,0.14,100$ are chosen such that the length of the minimizers satisfy $L(\gamma) / L \approx 4,2,1$, respectively. The density of the measure $\omega$ corresponds to the thickness of the curves.}
  \label{fig:L_alpha_experiment}
\end{figure}

\textbf{Influence of the degree.} Now, we like to show the effect of the truncation parameter $t$, i.e., the degree of $K_{t}$. From the previous observations and discussions we infer that the choice $N \sim L^{2}$ and $\alpha \sim L^{-5}$ are quite reasonable. In this experiment we shall provide some evidence that the choice $t \sim L$ might also be reasonable.
In Figure~\ref{fig:t_experiment} we depict the local minimizers of \eqref{eq:constant_speed_min_relaxed} for fixed parameters $L=2,4,8,16$, whith $\alpha = 0.2 \cdot L^{-5}$, $N=20 L^{2}$. We can observe that for $t = c L$ the corresponding local minimizers have common features. For instance, if $c=2$ ($t \approx L(\gamma)$) the minimizers have mostly vertical and horizantal line segments, similar to a path along a grid. Furthermore for sufficientt large $c$, say $c\ge 4$, the minimizers appear to provide {\color{blue} mercedes-star junctions (oder wie würdet ihr dazu sagen ;)}. 

\begin{figure}
  \begin{tabular}{ccccc}
$t=8$  &  $t=16$   & $t=32$  & $t=64$  & $t=128$ \\
  \includegraphics[width=0.18\textwidth]{Images/T8_L2} & 
  \includegraphics[width=0.18\textwidth]{Images/T16_L2} & 
  \includegraphics[width=0.18\textwidth]{Images/T32_L2} & 
  \includegraphics[width=0.18\textwidth]{Images/T64_L2} & 
  \includegraphics[width=0.18\textwidth]{Images/T128_L2} \\
$L(\gamma)\approx 4.07$ &
$L(\gamma)\approx 4.07$ &
$L(\gamma)\approx 4.06$ &
$L(\gamma)\approx 4.06$ &
$L(\gamma)\approx 4.05$\\ 
  \includegraphics[width=0.18\textwidth]{Images/T8_L4} & 
  \includegraphics[width=0.18\textwidth]{Images/T16_L4} & 
  \includegraphics[width=0.18\textwidth]{Images/T32_L4} & 
  \includegraphics[width=0.18\textwidth]{Images/T64_L4} & 
  \includegraphics[width=0.18\textwidth]{Images/T128_L4} \\
$L(\gamma)\approx 8.48$ &
$L(\gamma)\approx 8.28$ &
$L(\gamma)\approx 8.32$ &
$L(\gamma)\approx 8.23$ &
$L(\gamma)\approx 8.22$\\ 
  \includegraphics[width=0.18\textwidth]{Images/T8_L8} & 
  \includegraphics[width=0.18\textwidth]{Images/T16_L8} & 
  \includegraphics[width=0.18\textwidth]{Images/T32_L8} & 
  \includegraphics[width=0.18\textwidth]{Images/T64_L8} & 
  \includegraphics[width=0.18\textwidth]{Images/T128_L8} \\
$L(\gamma)\approx 10.43$ &
$L(\gamma)\approx 16.96$ &
$L(\gamma)\approx 16.77$ &
$L(\gamma)\approx 16.63$ &
$L(\gamma)\approx 16.4$\\ 
  \includegraphics[width=0.18\textwidth]{Images/T8_L16} & 
  \includegraphics[width=0.18\textwidth]{Images/T16_L16} & 
  \includegraphics[width=0.18\textwidth]{Images/T32_L16} & 
  \includegraphics[width=0.18\textwidth]{Images/T64_L16} & 
  \includegraphics[width=0.18\textwidth]{Images/T128_L16} \\
$L(\gamma)\approx 16.00$ &
$L(\gamma)\approx 20.88$ &
$L(\gamma)\approx 34.09$ &
$L(\gamma)\approx 33.52$ &
$L(\gamma)\approx 33.35$\\ 
  \end{tabular}
  \caption{Local minimizers  of \eqref{eq:constant_speed_min_relaxed} for $\mu$ the Lebesgue measure, which show the effect of the degree $t$ of the kernel $K_{t}$. Column-wise we increase $t=8,16,32,64,128$ and row-wise we increase $L=2,4,8,16$ where $\alpha = 0.2 L^{-5}$ and $N=20 L^{2}$. Note, the degree $t$ somehow steers the resolution of the curves. It appears that the spacing of the curves is bounded by $t^{-1}$.  Moreover the choice $t = c L$ leads to similar patterns. For instance, if $c=2$ the patterns are given mostly by horizontal or vertical line segments.}
  \label{fig:t_experiment}
\end{figure}


\subsection{Optimization Algorithms}
For optimization on Riemannian manifolds we refer to {\color{blue}[Todo: Refs]}, where the standard algorithms for constrained and unconstrained in Euclidean space are generalized to Riemannian manifolds. 

Our algorithm of choice is the nonlinear conjugate gradient (CG) method, cf. Algorithm~\ref{alg:cg}. This algorithm fits into the class of descent methods. More precisely, given a sufficient smooth function $f:\mathbb X \to \mathbb R$ and a starting point $x^{(0)} \in \mathbb X$, then a sequence of points $x^{(k)} \in \mathbb X$ is generated accordingly to the update rule
\begin{equation}
  \label{eq:descent_method}
  x^{(k+1)} := \gamma_{x^{(k)},d^{(k)}}(\alpha^{(k)}), \qquad k=0,1,\dots,
\end{equation}
where $d^{(k)}$ are descent directions, i.e.,
\[
  \langle \nabla_{\mathbb X} f(x^{(k)}), d^{(k)} \rangle <  0,
\]
and $\gamma_{x,d}:[0,\infty)\to \mathbb X$ is the geodesic curve satisfying
\begin{equation}
  \label{eq:geodesic_descent}
  \gamma_{x,d}(0) = x, \qquad \dot \gamma_{x,d}(0) = d.
\end{equation}
For an appropriate choice of descent directions $d^{(k)}$ and step lengths $\alpha^{(k)}$, the sequence of points $x^{(k)}$ defined by  \eqref{eq:descent_method} converges towards a local minimizer $x_{*}$ of $f$.

The advantage of the CG method is its simplicity together with fast convergence at low computational cost. 
Especially for smooth high dimensional unconstrained optimization problems it has proven very efficient, cf. {\color{blue}[TODO: Referenzen]}. 
Indeed, under suitable assumptions the sequence produced by Algorithm~\ref{alg:cg} converges superlinearly ($N\cdot\dim(\mathbb X)$-step quadratically) 
towards a local minimum, cf. {\color{blue} [Diss, Section 3.3.2 and Theorem 3.27.]}.


\begin{algorithm}[ht]
  \caption{(\textbf{ CG Method with Restarts})}
  \begin{algorithmic}
    \State \textbf{Parameters:} maximal iterations $k_{\max} \in \mathbb N$
    \State \textbf{Input:} twice differentiable function $f:\mathbb X^{N} \to [0,\infty)$,
    initial point $x^{(0)} \in \mathbb X^{N}$
    \State \textbf{Initialization:} $g^{(0)} := \nabla_{\mathbb X^{N}}f(x^{(0)})$, 
    $d^{(0)}:= - g^{(0)}$, $r:=0$
    \For{$k:=0,\dots,k_{\max}$}
%    \State $\alpha^{(k)}_{0} :=
%    \begin{cases}
%      \left| \frac{ \langle d^{(k)}, g^{(k)} \rangle}{ \langle d^{(k)}, \mathrm{H}_{\mathbb X^{N}}f(x^{(k)}) d^{(k)}\rangle } \right|,
%      & \langle d^{(k)}, \mathrm{H}_{\mathbb X^{N}}f(x^{(k)}) d^{(k)} \rangle \ne 0,\\
%      1, & \text{ else}
%    \end{cases}
%    $
%   % $ \alpha^{(k)}:= \mathrm{min} \big\{ \frac{\mathrm d}{\mathrm d t} f \circ \gamma_{x^{(k)},d^{(k)}}(t) = 0 \big\}$  (perform a line search)
    \State $x^{(k+1)} := \gamma_{x^{(k)}, d^{(k)}}(\alpha^{(k)})$  where $\alpha^{(k)}$ is determined by Algorithm~\ref{alg:armijo}
 
    \State
    $
    \tilde d^{(k)}:=  \dot \gamma_{x^{(k)}, d^{(k)}}(\alpha^{(k)})
    %\in \Ts_{\zb  x^{(k+1)}}\M.
    $
    \State  $g^{(k+1)} := \nabla_{\mathbb X^{N}}f(x^{(k+1)})$
    \State
    $
    \beta^{(k)}  :=  
    \begin{cases}
    \frac{ \langle \tilde d^{(k)}, \mathrm H_{\mathbb X^{N}} f(x^{(k+1)}) g^{(k+1)}\rangle }{\langle \tilde d^{(k)}, \mathrm H_{\mathbb X^{N}} f(x^{(k+1)}) \tilde
        d^{(k)} \rangle},
      & \langle \tilde d^{(k}, \mathrm H_{\mathbb X^{N}}f(x^{(k+1)}) \tilde d^{(k)} \rangle \ne 0,\\
      0, & \text{ else}
    \end{cases}
    $
    \State $d^{(k+1)}:=- g^{(k+1)} + \beta^{(k)} \tilde d^{(k)}$
    \If{$\langle d^{(k+1)}, g^{(k+1)} \rangle > 0$ or $(k+1) \equiv r \mod N \mathrm{dim}(\mathbb X)$}
    \State $d^{(k+1)} = - g^{(k+1)}$
    \State $r := k+1$
    \EndIf
    \EndFor
    \State \textbf{Output:} iteration sequence $x^{(0)},x^{(1)}, \dots \in \mathbb X^{N}$
\end{algorithmic}
\label{alg:cg}
\end{algorithm}

%Unfortunately, the determination of the exact step size $\alpha^{(k)}$ in  Algorithm~\ref{alg:cg} is practically prohibited. 
%Therefore, several conditions have been proposed to guarantee the convergence towards a local minimum. We use the Armijo condition implemented in Algorithm~\ref{alg:armijo}.

\begin{algorithm}[ht]
\caption{(\textbf{Armijo Line Search})}
\begin{algorithmic}
\State \textbf{Parameters:} $0 < \mu < \tfrac12$, $0 < \tau < 1$, maximal iterations $k_{\max} \in \mathbb N$
\State \textbf{Input:} smooth function $f:\mathbb X^{N} \to [0,\infty)$, starting point $x \in \mathbb X^{N}$, descent direction $d \in \mathrm T_{x}\mathbb X^{N}$
\State \textbf{Initialization:}
\[\alpha^{(0)} :=
    \begin{cases}
      \left| \frac{ \langle d, \nabla_{\mathbb X^{N}} f(x) \rangle}{ \langle d, \mathrm{H}_{\mathbb X^{N}}f(x) d\rangle } \right|,
      & \langle d, \mathrm{H}_{\mathbb X^{N}}f(x) d \rangle \ne 0,\\
      1, & \text{ else}
    \end{cases}
  \]
\State $k:=0$
\While{$f \circ \gamma_{x,d} (\alpha^{(k)}) - f(x) \ge \alpha^{(k)} \mu \langle \nabla_{\mathbb X^{N}} f(x), d \rangle$ and $k < k_{\max}$}
\State $\alpha^{(k+1)} := \tau \alpha^{(k)}$
\State $k:=k+1$
\EndWhile
\State \textbf{Output:} $\alpha^{(k)}$ (success if $k \le k_{\max}$)
\end{algorithmic}
\label{alg:armijo}
\end{algorithm}

\subsection{Implementation Details}

We briefly summarize the main ingredients of our numerical implementation.

For every  use-case $\mathbb X \in \{ \mathbb T^{d}, \mathbb S^{2}, \mathrm{SO(3)}, \mathcal G_{2,4}\}$ 
we provide a local parameterization $h:\Omega \to \mathbb X$ with coordinates $y \in \Omega \subset \mathbb R^{d}$ 
sucht that $h(\overline \Omega) = \mathbb X$. {\color{blue} The case $\mathbb X^{N}$ 
is handled straightforward by considering $h_{N}:\Omega^{N} \to \mathbb X^{N}$ given by $h_{N}(y_{1},\dots,y_{N}) := (h(y_{1}),\dots,h(y_{N}))$.}

Then for $y \in \Omega$ the Jacobian matrix
\[
  \mathrm D h(y) := \Big(  \frac{\partial}{\partial y_{1}} h(y), \dots, \frac{\partial}{\partial y_{d}} h(y) \Big)
\]
is non-singular. Thus, the vectors
\[
  e_{i}(y) := \frac{\partial}{\partial y_{i}} h(y), \qquad i=1,\dots,d,
\]
form a basis of the tangent space $\mathrm T_{h(y)}\mathbb X$ and the local parameterization $h$ induces the Riemannian matrix
\[
  G_{h}(y) :=  \mathrm D h(y)^{\top} \mathrm D h(y)
\]
Moreover, given a vector $v \in \mathrm T_{h(y)}\mathbb X$ by
\[
  v = \mathrm D h(y) w, \qquad w \in \mathbb R^{d}
\]
we can recover the coordinates $w$ by
\[
 w =   G_{h}^{-1}(y) \mathrm D h(y)^{\top} v.
\]
Let a smooth function $f:\mathbb X \to \mathbb R$ be given. Then the gradient of $f$ is given in local coordinates as
\[
  \nabla_{h} (f \circ h) (y) := G^{-1}_{h}(y) \nabla (f\circ h)(y),
\]
where
\[
  \nabla (f\circ h)(y) := \Big(\frac{\partial}{\partial y_{1}} (f\circ h)(y),\dots,\frac{\partial}{\partial y_{d}} (f\circ h)(y)\Big),
\]
is the usual gradient of $f\circ h$. This definition leads to the tangent vector 
\[
 \nabla_{\mathbb X}f (x) = \mathrm D h(y) \nabla_{h} (f\circ h) (y), \qquad x=h(y).
\]
The Hessian matrix of $f$ is given in local coordinates by
\[
  \mathrm H_{h} (f\circ h) (y) := \mathrm H (f \circ h) (y) - \mathrm N (f\circ h) (y)
\]
where
\[
  \mathrm H (f \circ h) (y) := \Big(\frac{\partial^{2}}{\partial y_{i} \partial y_{j}} (f\circ h)(y)\Big)_{i,j=1}^{d}
\]
is the usual Hessian of $f \circ h$ and
\[
   \mathrm N (f\circ h) (y) = \sum_{m=1}^{d} \big(\nabla (f\circ h)(y)\big)_{m} \Gamma_{h}^{m}(y)
\]
is a matrix encoding the normal derivatives using the Christoffel matrices
\[
  \Gamma_{h}^{m}(y) :=  \frac12  \Big( \sum_{l=1}^{d} \Big(\frac{\partial}{\partial y_{i}}\big(G_{h}(y)\big)_{j,l} + \frac{\partial}{\partial y_{j}}\big(G_{h}(y)\big)_{i,l} - \frac{\partial}{\partial y_{l}}\big(G_{h}(y)\big)_{i,j}\Big) \big(G_{h}^{-1}(y)\big)_{l,m} \Big)_{i,j=1}^{d}.
\]
Then the multiplication of the Hessian of $f$ in the tangent space reads as
\[
\big( \mathrm H_{\mathbb X} f (x) \big) v = \mathrm D h(y) G_{h}^{-1}(y) \mathrm H_{h} (f\circ h) (y) \mathrm G_{h}^{-1}(y) \mathrm D h(y)^{\top} v, \qquad v \in \mathrm T_{x}\mathbb X, \qquad x=h(y).
\]
{\color{blue} For the approximation of the usual Hessian we shall use the finite difference quotient
\[
  \mathrm H(f \circ h)(y) \approx \frac{\|w\|_{2}}{t} \Big( (f \circ h)\big(y + \tfrac{t}{\|w\|_{2}} w\big) - (f\circ h)(y)\Big) 
\]
in our algorithms, with $t=10^{-8}$.}


Finally, we shall provide formulas for the geodesics $\gamma_{x,d}:[0,\infty)\to \mathbb X$ with starting point $x \in \mathbb X$ and direction $d \in \mathrm T_{x}\mathbb X$, cf. \eqref{eq:geodesic_descent}. Note, that we can recover the local coordinates from $y=h^{-1}(x)$.  


\textbf{Torus.}
%The torus $\mathbb T^{d}$ is embedded in $\mathbb R^{2d}$ by the local parameterization 
%\begin{equation}
%  \label{eq:Td_param}
%  h(\alpha) = %(\sin(\alpha_{1}),\cos(\alpha_{1}),\dots,\sin(\alpha_{d}),\cos(\alpha_{d})), %\qquad \alpha := (\alpha_{1},\dots,\alpha_{d}) \in \mathbb R^{d}.
%\end{equation}
%From $h(\alpha)=h(\alpha + 2\pi k)$, $k \in \mathbb Z^{d}$, we infer $\mathbb T^{d} \cong \mathbb R^{d} / \mathbb Z^{d}$. Hence, the torus $\mathbb T^{d}$ can be considered as Euclidean space $\mathbb R^{d}$ where points are $2\pi$-periodically identified. Therefore, we shall forget about the actual parameterization \eqref{eq:Td_param} and perform any geometric operation in local coordinates, i.e., in  Euclidean space $\mathbb R^{d}$. 
Since $\mathbb T^{d} \cong \mathbb R^{d} / \mathbb Z^{d}$  we can apply the standard optimization methods for Euclidean space $\mathbb R^{d}$. That means we use the local parameterization $h:\mathbb R^{d} \to \mathbb R^{d}$ with $h(y) = y$ and the geodesic with starting point $x \in \mathbb R^{d}$ and direction $d \in \mathbb R^{d}$ is given by
\[
  \gamma_{x,d}(t) = x + t d,  \qquad t \in [0,\infty).
\]
The function
\[
  E_{\mu}(x_{1},\dots,x_{N}) = \sum_{l \in ([-t,t] \cap \mathbb Z)^{d}} \lambda_{l} \Big| \mu_{l} - \frac 1N \sum_{k=1}^{N} \mathrm{e}^{2\pi \mathrm i \langle x_{k}, l\rangle  }\Big|^{2}
\]
and its gradient $\nabla E_{\mu}$
%\[
%  \nabla E_{\mu}(x_{1},\dots,x_{N}) = \big(\tfrac{\partial}{\partial x_{i,j}} E_{\mu}(x_{1},\dots,x_{N}) %\big)_{i=1,\dots,N; j=1,\dots,d}
%\]
%and the multiplication of a vector $v \in \mathbb R^{N d}$ with the Hessian
%\[
%  \mathrm{H}_{(\mathbb R^{d})^{N}} E_{\mu}(x_{1},\dots,x_{N}) v = \big( %\sum_{m=1}^{N}\sum_{n=1}^{d} v_{m,n} \tfrac{\partial^{2}}{\partial %x_{i,j}\partial x_{m,n}} E_{\mu}(x_{1},\dots,x_{N}) \big)_{i=1,\dots,N; %j=1,\dots,d}
%\]
can be evaluated in $\mathcal O(t^{d} \log(t) + N)$ arithmetic operations, cf. {\color{blue}[Diss, Section 5.2.1]}.

The penalty term
\[
  F_{\alpha,L}(x_{1},\dots,x_{N}) = \alpha \sum_{i=1}^{N}  \big(\|x_{k-1}-x_{k}\|_{2} - \tfrac{L}{N}\big)^{2}
\]
and its gradient $\nabla F_{\alpha,L}$ 
%\[
%\nabla F_{\alpha,L}(x_{1},\dots,x_{N}) = \big(\tfrac{\partial}{\partial x_{i,j}} %F_{\alpha,L}(x_{1},\dots,x_{N}) \big)_{i=1,\dots,N; j=1,\dots,d}
%\]
can be evaluated in $\mathcal O(N)$.

\textbf{Sphere.} The sphere $\mathbb S^{2}$ is parameterized by
\[
  h(\theta,\varphi) = (\sin(\theta) \sin(\varphi), \sin(\theta) \cos(\varphi), \cos(\theta)), \qquad (\theta,\varphi) \in [0,\pi]\times [0,2\pi).
\]
Restricting the domain of $h$ to $\Omega := (0,\pi)\times (0,2\pi)$ we have a local parameterization with $h(\overline \Omega) = \mathbb S^{2}$ and inverse
\[
  h^{-1}(x) :=  \begin{pmatrix}
    \arccos(x_{3}) \\
    2 \arctan \Big( x_{2}  \Big(x_{1} - \sqrt{x_{1}^{2}+x_{2}^{2}}\Big)^{-1} \Big) + \pi
  \end{pmatrix}
  , \qquad x \in h(\Omega).
\]
The geodesic with starting point $x \in \mathbb S^{2}$ and direction $d \in \mathrm T_{x}\mathbb S^{2}$ is given by
\[
  \gamma_{x,d}(t) := \cos(t \|d\|_{2}) x + \sin(t \|d\|_{2}) \tfrac{d}{\|d\|_{2}}, \qquad t \in [0,\infty).
\]

The function
\[
  E_{\mu}(\theta_{1},\varphi_{1},\dots,\theta_{N},\varphi_{N}) = \sum_{m=0}^{t} \sum_{n=-m}^{m} \lambda_{m}^{n} \Big| \mu_{m}^{n} - \frac 1N \sum_{k=1}^{N} Y_{m}^{n}(\theta_{k},\varphi_{k})\Big|^{2}
\]
and its gradient $\nabla E_{\mu}$ can be evaluated in $\mathcal O(t^{2} \log^{2}(t) + N)$ arithmetic operations, cf. {\color{blue}[Diss, Section 5.2.2]}.

The penalty term
\[
  F_{\alpha,L}(\theta_{1},\varphi_{1},\dots,\theta_{N},\varphi_{N}) = \alpha \sum_{i=1}^{N} \big(  \mathrm{dist}(h(\theta_{k-1},\varphi_{k-1}),h(\theta_{k},\varphi_{k})) - \tfrac{L}{N}\big)^{2}
\]
with
\[
  \mathrm{dist}(h(\theta_{k-1},\varphi_{k-1}),h(\theta_{k},\varphi_{k})) = \arccos \Big( \cos(\theta_{k-1}) \cos(\theta_{k}) - \cos(\varphi_{k-1}-\varphi_{k}) \sin(\theta_{k-1})\sin(\theta_{k})\Big)
\]
and its gradient $\nabla F_{\alpha,L}$ can be evaluated in $\mathcal O(N)$.

\textbf{Rotation Group.} The rotation group $\mathrm{SO(3)}$ is parameterized by
\[
  h(\varphi,\theta,\phi) = R(e_{3},\varphi) R(e_{2},\theta) R(e_{3},\phi), \qquad (\varphi,\theta,\phi) \in [0,2\pi)\times [0,\pi] \times [0,2\pi),
\]
where 
\[
  R(r,\omega) = (1-\cos(\omega))rr^{\top} + 
\begin{pmatrix}
 \cos(\omega) & - r_{3} \sin(\omega) & r_{2} \sin(\omega) \\
r_{3} \sin(\omega) & \cos(\omega) & -r_{1} \sin(\omega) \\
- r_{2} \sin(\omega) &  r_{1} \sin(\omega) & \cos(\omega)  \\
\end{pmatrix}
\]
is the rotation matrix with rotation axis $r$ and rotation angle $\omega$. Restricting the domain of $h$ to $\Omega := (0,2\pi) \times (0,\pi) \times (0,2\pi)$ we have a local parameterization with $h(\overline \Omega)= \mathrm{SO(3)}$ and inverse
\[
  h^{-1}(R) = 
\begin{pmatrix}
  2 \arctan( R_{2,3} \Big(R_{1,3} - \sqrt{R_{1,3}^{2}+R_{2,3}^{2}}\Big)^{-1} \Big) + \pi \big) \\
  \arccos(R_{3,3})\\
  2 \arctan( R_{3,2} \Big(- R_{3,1} - \sqrt{R_{3,1}^{2}+R_{3,2}^{2}}\Big)^{-1} \Big) + \pi \big) \\
\end{pmatrix}, \qquad R \in h(\Omega).
\]
The geodesic with starting point $R \in \mathrm{SO(3)}$ and direction $D \in \mathrm T_{x}\mathrm{SO(3)}$ is given by
\[
  \gamma_{R,D}(t) :=  R \exp( t R^{\top} V), \qquad t \in [0,\infty),
\]
where $\exp:\mathbb R^{3\times 3} \to \mathbb R^{3\times 3}$ is the matrix exponential defined by
\[
  \exp(A) := \sum_{k=0}^{\infty} \frac{A^{k}}{k!}.
\]

The function
\[
  E_{\mu}(\varphi_{1},\theta_{1},\phi_{1},\dots,\varphi_{N},\theta_{N},\phi_{N}) = \sum_{m=0}^{t} \sum_{n=-m}^{m} \sum_{n'=-m}^{m}\lambda_{m}^{n,n'} \Big| \mu_{m}^{n,n'} - \frac 1N \sum_{k=1}^{N} D_{m}^{n,n'}(\varphi_{k},\theta_{k},\phi_{k})\Big|^{2}
\]
and its gradient $\nabla E_{\mu}$ can be evaluated in $\mathcal O(t^{3} \log^{2}(t) + N)$ arithmetic operations, cf. {\color{blue}[Diss, Section 5.2.3]}.

The penalty term
\[
  F_{\alpha,L}(\varphi_{1},\theta_{1},\phi_{1},\dots,\varphi_{N},\theta_{N},\varphi_{N}) = \alpha \sum_{i=1}^{N} \big(  \mathrm{dist}(h(\varphi_{k-1},\theta_{k-1},\phi_{k-1}),h(\varphi_{k},\theta_{k},\phi_{k})) - \tfrac{L}{N}\big)^{2}
\]
with
\[
  \begin{aligned}
    & \mathrm{dist}(h(\varphi_{k-1},\theta_{k-1},\phi_{k-1}),h(\varphi_{k},\theta_{k},\phi_{k})) \\
    =&  {\tiny 2 \arccos\left| \cos\Big(\tfrac{\varphi_{k-1}-\varphi_{k}}{2}\Big) \cos\Big(\tfrac{\theta_{k-1}-\theta_{k}}{2}\Big)  \cos\Big(\tfrac{\phi_{k-1}-\phi_{k}}{2}\Big) - \sin\Big(\tfrac{\varphi_{k-1}-\varphi_{k}}{2}\Big) \cos\Big(\tfrac{\theta_{k-1}+\theta_{k}}{2}\Big)  \sin\Big(\tfrac{\phi_{k-1}-\phi_{k}}{2}\Big) \right|}
\end{aligned}
\]
and its gradient $\nabla F_{\alpha,L}$ can be evaluated in $\mathcal O(N)$.

\textbf{Grassmannian.} Since $\mathcal G_{2,4} \cong \mathbb S^{2} \times \mathbb S^{2} / \{\pm 1\}$ we can apply the optimization methods for the sphere $\mathbb S^{2}$. That means, let $\tilde h:[0,\pi]\times [0,2\pi) \to \mathbb S^{2}$ be the parameterization of the sphere $\mathbb S^{2}$, cf. [TODO],  we parameterize the double sphere $\mathbb S^{2} \times \mathbb S^{2}$ by

\[
%  h(\theta_{1},\varphi_{1},\theta_{2},\varphi_{2}) = \begin{pmatrix}
%    \sin(\theta_{1}) \sin(\varphi_{1}) \\
%    \sin(\theta_{1}) \cos(\varphi_{1}) \\
%    \cos(\theta_{1})\\
%    \sin(\theta_{2}) \sin(\varphi_{2}) \\
%    \sin(\theta_{2}) \cos(\varphi_{2}) \\
%    \cos(\theta_{2})
%  \end{pmatrix}
   h(\theta_{1},\varphi_{1},\theta_{2},\varphi_{2}) = (\tilde h(\theta_{1},\varphi_{1}), \tilde h(\theta_{2},\varphi_{2})),  
  , \qquad (\theta_{1},\varphi_{1},\theta_{2},\varphi_{2}) \in ([0,\pi]\times [0,2\pi))^{2}.
\]
Restricting the domain of $h$ to $\Omega := ((0,\pi)\times (0,2\pi))^{2}$ we have a local parameterization with $h(\overline \Omega) = \mathbb S^{2}\times \mathbb S^{2}$ and inverse
\[
  h^{-1}(x) :=
  %\begin{pmatrix}
  %  \arccos(x_{3}) \\
  %  2 \arctan \Big( x_{2}  \Big(x_{1} - \sqrt{x_{1}^{2}+x_{2}^{2}}\Big)^{-1} \Big) + \pi\\
  %  \arccos(x_{6}) \\
  %  2 \arctan \Big( x_{5}  \Big(x_{4} - \sqrt{x_{4}^{2}+x_{5}^{2}}\Big)^{-1} \Big) + \pi\\
  %\end{pmatrix}
    \begin{pmatrix}
      \tilde h^{-1}(x_{1})\\
      \tilde h^{-1}(x_{2})
    \end{pmatrix}
  , \qquad x = (x_{1},x_{2}) = \in h(\Omega).
\]
The geodesic with starting point $x=(x_{1}, x_{2}) \in \mathbb S^{2} \times \mathbb S^{2}$ and direction $d = (d_{1},d_{2}) \in \mathrm T_{x_{1}}\mathbb S^{2} \times \mathrm T_{x_{2}}\mathbb S^{2}$ is given by
\[
  \gamma_{x,d}(t) :=
%  \begin{pmatrix}
 (   \cos(t \|d_{1}\|_{2}) x_{1} + \sin(t \|d_{1}\|_{2}) \tfrac{d_{1}}{\|d_{1}\|_{2}},
    \cos(t \|d_{2}\|_{2}) x_{2} + \sin(t \|d_{2}\|_{2}) \tfrac{d_{2}}{\|d_{2}\|_{2}})
    % \end{pmatrix}
    , \qquad t \in [0,\infty).
\]

The function
\[
  \begin{aligned}
    &E_{\mu}(\theta_{1},\varphi_{1},\dots,\theta_{2N},\varphi_{2N}) \\
    = &\sum_{m_{1},m_{2}=0}^{t} \sum_{n_{1}=-m_{1}}^{m_{1}} \sum_{n_{2}=-m_{2}}^{m_{2}} \lambda_{m_{1},m_{2}}^{n_{1},n_{2}} \Big| \mu_{m_{1},m_{2}}^{n_{1},n_{2}} - \frac 1N \sum_{k=1}^{N} Y_{m_{1}}^{n_{1}}(\theta_{2k-1},\varphi_{2k-1}) Y_{m_{2}}^{n_{2}}(\theta_{2k},\varphi_{2k})\Big|^{2}
  \end{aligned}
\]
and its gradient $\nabla E_{\mu}$ can be evaluated in $\mathcal O(t^{4} \log^{2}(t) + N)$ arithmetic operations, cf. {\color{blue}[Discrepanz paper]}.

The penalty term
\[
  F_{\alpha,L}(\theta_{1},\varphi_{1},\dots,\theta_{2N},\varphi_{2N}) = \alpha \sum_{i=1}^{N} \big(  \mathrm{dist}(h(\theta_{2k-3},\varphi_{2k-3},\theta_{2k-2},\varphi_{2k-2}),h(\theta_{2k-1},\varphi_{2k-1},\theta_{2k},\varphi_{2k})) - \tfrac{L}{N}\big)^{2}
\]
with
\[\begin{aligned}
    &\mathrm{dist}(h(\theta_{2k-3},\varphi_{2k-3},\theta_{2k-2},\varphi_{2k-2}),h(\theta_{2k-1},\varphi_{2k-1},\theta_{2k},\varphi_{2k}))\\
    = & \Big( \arccos \Big( \cos(\theta_{2k-3}) \cos(\theta_{2k-1}) - \cos(\varphi_{2k-3}-\varphi_{2k-1}) \sin(\theta_{2k-3})\sin(\theta_{2k-1})^{2}\Big)\\
   & +   \arccos \Big( \cos(\theta_{2k-3}) \cos(\theta_{2k-1}) - \cos(\varphi_{2k-3}-\varphi_{2k-1}) \sin(\theta_{2k-3})\sin(\theta_{2k-1})\Big)^{2}\Big)^{\frac12}
  \end{aligned}
\]
and its gradient $\nabla F_{\alpha,L}$ can be evaluated in $\mathcal O(N)$.


\subsection{Finding good initial points/curves}

Since the objective function in \eqref{eq:constant_speed_min} is highly non-convex, the main problem is to find good initial points $x_{k} \in \mathbb X$.

Our heuristic is as follows. We start with a curve $\gamma:[0,1]\to \mathbb X$ of small length $L(\gamma)$ and solve the problem \eqref{eq:constant_speed_min_relaxed} for certain $\alpha$ in dependency of the parameter $L$.

\textbf{2d-torus.}
We choose the following dependencies $\alpha = L^{-4}$, $N =11 L^{2}$, $t=3 L$. We start with a circle of radius $r=\frac13$ and solve \eqref{eq:constant_speed_min_relaxed} for $L=3$ using the conjugate gradient method. The obtained minimizer is used as initial point for the next minimization problem, where the parameter $L$ is doubled. The procedure is stopped when the twice the length of the minimizer bigger than the prescribed length.

\section{Curves integrating polynomials on $\mathbb S^{d}$} \label{sec:intro}

We are going to construct a curve $\gamma_{N}:[0,1]\to \mathbb S^{d}$ of constant speed $\sim N^{d-1}$ and a probability measure $\omega_{N}$ on the interval $[0,1]$, such that the weighted push-forward  $(\gamma_{N})_{*} \mathrm d \omega_{N}$ integrates exactly all polynomials of degree $N$ on the sphere $\mathbb S^{d}$, i.e.,
\begin{equation}
  \label{eq:quad_curve_Sd}
  \int_{\mathbb S^{d}} p(x) \mathrm d \sigma_{\mathbb S^{d}}(x) = \int_{0}^{1} p( \gamma_{N}(t)) \mathrm d \omega_{N}(t), \qquad p \in \mathrm{Pol}_{N}(\mathbb S^{d}). 
\end{equation}
Recall, that the integral of a continuous function $f:\mathbb S^{d}\to \mathbb C$ can be computed by integrating inductively along the coordinates, i.e.,
\[
  \int_{\mathbb S^{d}} f(x) \mathrm d \sigma_{\mathbb S^{d}}(x) = \int_{0}^{\pi} c_{d} \sin(\theta)^{d-1} \int_{\mathbb S^{d-1}} f(\cos(\theta),\sin(\theta)\tilde x) \mathrm d \sigma_{\mathbb S^{d-1}}(\tilde x)   \mathrm d \theta,
\]
where $c_{d} := \big( \int_{0}^{\pi} \sin(\theta)^{d-1} \mathrm d \theta \big)^{-1} $. Note the measures $\sigma_{\mathbb S^{d}}$ are normalized.

Now, let us consider a quadrature formula on the sphere $\mathbb S^{d-1}$ with  nodes  $\tilde x_{i} \in \mathbb S^{d-1}$ and weights $a_{i} > 0$, $i=1,\dots, M \sim N^{d-1}$, i.e.,
\[
  \int_{\mathbb S^{d-1}} p(x) \mathrm d \sigma_{\mathbb S^{d-1}}(x) = \sum_{i=1}^{M} a_{i} p(\tilde x_{i}), \qquad p \in \mathrm{Pol}_{N}(\mathbb S^{d-1}).   
\]
Then we define $\gamma_{N}:[0,1]\to\mathbb S^{d}$ for $t \in [(i-1)/M,i/M]$, $i=1,\dots,M$, by
\[
  \gamma_{N}(t) := \gamma_{N,i}(2\pi M t) 
\]
where
\[
  \gamma_{N,i}(\alpha) := (\cos(\alpha), \sin(\alpha) \tilde x_{i}), \quad \alpha \in [0,2\pi],\qquad ,i=1,\dots,M.
\]
The curve is well-defined and closed, since we have $\gamma_{N,1}(0) = (1,0,\dots,0) = \gamma_{N,i}(2\pi)  = \gamma_{N,i+1}(0) = \gamma_{N,i}(2\pi)$, $j=1,\dots,M-1$.  Furthermore, it is straightforward to check, that $\gamma_{N}(t)$ has constant speed
\[
 \| \dot \gamma_{N}(t)\|_{2} = \| \dot \gamma_{N,i}(2 \pi M t)\|_{2}  =2\pi M \sim N^{d-1}, \qquad i=1,\dots,M.
\]
Additionally, we define the density $w_{N}:[0,1]\to \mathbb \R_{+}$  for $t \in [(i-1)/M,i/M]$, $i=1,\dots,M$, by
\[
  w_{N}(t) := w_{N,i}(2\pi M t) 
\]
where
\[
  w_{N,i}(\alpha) := a_{i} c_{d}  \pi  M|\sin(\alpha)|^{d-1}, \quad \alpha \in [0,2\pi],\qquad i=1,\dots,M.
\]
The density is well-defined, since $w_{N,1}(0) = 0 = w_{N,i}(2\pi) = w_{N,i+1}(0) = w_{N,M}(2\pi)$, $i=1,\dots,M-1$. Moreover, $w_{N}$ is by definition of the constant $c_{d}$ and weights $a_{i}$, indeed a probability density
\[
  \begin{aligned}
    \int_{0}^{1} w_{N}(t) \mathrm d t &  = \sum_{i=1}^{M} \int_{\frac{i-1}{M}}^{\frac{i}{M}}  w_{N,i}(2\pi M t) \mathrm d t =
    \frac{1}{2\pi M}  \sum_{i=1}^{M} \int_{0}^{2\pi}  w_{N,i}(\alpha) \mathrm d \alpha = 
   c_{d} \frac{1}{2} \sum_{i=1}^{M} a_{i} \int_{0}^{2\pi}  |\sin(\theta)|^{d-1} \mathrm d \theta = 1,
   \end{aligned}
 \]
 where we substitute in the second equation $\alpha =2 \pi M t$. Similarly, for $p \in \mathrm{Pol}_{N}(\mathbb S^{d})$ we verify 
\[
  \begin{aligned}
    \int_{0}^{1} p(\gamma_{N}(t)) w_{N}(t) \mathrm d t
    =&  \sum_{i=1}^{M} \int_{\frac{i-1}{M}}^{\frac i M} p(\gamma_{N,i}(t)) w_{N,i}(t) \mathrm dt \\
    =&  \sum_{i=1}^{M} \int_{\frac{i-1}{2M}}^{\frac i M} p(\gamma_{N,i}(2 \pi M t)) w_{N,i}(2\pi M t) \mathrm dt \\
    =&   \int_{0}^{2\pi}  \frac{1}{2\pi M} \sum_{i=1}^{M}  p(\gamma_{N,i}(\alpha)) w_{N,i}(\alpha) \mathrm d \alpha\\
%    \int_{0}^{2\pi}  \frac{1}{2\pi M} \sum_{i=1}^{M}  p(\gamma_{N,i}(\alpha)) w_{N,i}(\alpha) \mathrm d \alpha
    =  & \frac12\int_{0}^{2\pi}  c_{d} |\sin(\alpha)|^{d-1} \sum_{i=1}^{M} a_{i}  p(\cos(\alpha),\sin(\alpha)\tilde x_{i}) \mathrm d \alpha \\
    =  & \frac12\int_{0}^{\pi}  c_{d} |\sin(\alpha)|^{d-1} \sum_{i=1}^{M} a_{i}  p(\cos(\alpha),\sin(\alpha)\tilde x_{i}) \mathrm d \alpha \\
    +  & \frac12\int_{0}^{\pi}  c_{d} |\sin(\alpha+\pi)|^{d-1} \sum_{i=1}^{M} a_{i}  p(\cos(\alpha+\pi),\sin(\alpha+\pi)\tilde x_{j}) \mathrm d \alpha \\
    =  & \frac12\int_{0}^{\pi}  c_{d} |\sin(\alpha)|^{d-1} \sum_{i=1}^{M} a_{i}  p(\cos(\alpha),\sin(\alpha)\tilde x_{i}) \mathrm d \alpha \\
    +  & \frac12\int_{0}^{\pi}  c_{d} |\sin(\alpha)|^{d-1} \sum_{i=1}^{M} a_{i}  p(-\cos(\alpha),-\sin(\alpha)\tilde x_{i}) \mathrm d \alpha \\
  \end{aligned}
\]
Without loss of generality we can assume that $p$ is a homogeneous polynomial of degree $n \le N$, i.e., $p(r x) =r^{n} p(x).$ Hence, we arrive at
\[
  \begin{aligned}
    \int_{0}^{1} p(\gamma_{N}(t)) w_{N}(t) \mathrm d t
   =  & \frac12\int_{0}^{\pi}  c_{d} |\sin(\alpha)|^{d-1} \sum_{i=1}^{M} a_{i}  p(\cos(\alpha),\sin(\alpha)\tilde x_{i}) \mathrm d \alpha \\
    + & (-1)^{n} \frac12\int_{0}^{\pi}  c_{d} |\sin(\alpha)|^{d-1} \sum_{i=1}^{M} a_{i}  p(\cos(\alpha),\sin(\alpha)\tilde x_{i}) \mathrm d \alpha \\
  \end{aligned} 
\]
Using that the nodes $\tilde x_{i}$ and weights $w_i$ form a quadrature and that for fixed $\alpha \in [0,2\pi]$ the function $\tilde x \mapsto p(\cos(\alpha), \sin(\alpha) \tilde x)$ is a polynomial of degree at most $N$ on $\mathbb S^{d-1}$, we arrive at
\[
  \begin{aligned}
    \int_{0}^{1} p(\gamma_{N}(t)) w_{N}(t) \mathrm d t
   =  & \frac{1+(-1)^{n}}{2}\int_{0}^{\pi}  c_{d} |\sin(\alpha)|^{d-1} \int_{\mathbb S^{d-1}}  p(\cos(\alpha),\sin(\alpha)\tilde x) \mathrm d\sigma_{\mathbb S^{d-1}}(x) \mathrm d \alpha. 
  \end{aligned}
\]
Now, the assertion \eqref{eq:quad_curve_Sd} follows from 
\[
  \int_{\mathbb S^{d}} p(x) \mathrm d \sigma_{\mathbb S^{d}}(x) =
  \begin{cases}
    0, &  n \equiv 1 \mod 2,\\
  \int_{0}^{\pi}  c_{d} |\sin(\alpha)|^{d-1} \int_{\mathbb S^{d-1}}  p(\cos(\alpha),\sin(\alpha)\tilde x) \mathrm d\sigma_{\mathbb S^{d-1}}(x), & n \equiv 0 \mod 2.     
  \end{cases}
\]

\section{Curves integrating polynomials on $\mathrm{SO(3)}$} \label{sec:intro}

% ------------------------------------------------------------------------
We are going to construct a curve $\gamma_{N}:[0,1] \to \mathrm{SO(3)}$ of constant speed $N^{2}$ and a probability measure $\omega_{N}$ on the interval $[0,1]$, such that the weighted push-forward $(\gamma_{N})_{*}\mathrm d \omega_{N}$ integrates exactly all polynomials of degree $N$ on the rotation group $\mathrm{SO(3)}$, i.e.,
\begin{equation}
   \label{eq:quad_curve_Sd}
  \int_{\mathrm{SO(3)}} p(x) \mathrm d\sigma_{\mathrm{SO(3)}}(x) = \int_{0}^{1} p(\gamma_{N}(t)) \mathrm d\omega_{N}(t), \qquad p \in \mathrm{Pol}_{N}(\mathrm{SO(3)}).
\end{equation}

We shall use the well-known fact that the sphere $\mathbb S^{3}$ is a double covering of $\mathrm{SO(3)}$. That is, there is a surjective two-to-one mapping $r: \mathbb S^{3} \to \mathrm{SO(3)}$ satisfying $r(x) = r(-x)$, $x \in \mathbb S^{3}$. Moreover, we know from {\color{blue}[cite: A unified approach to scattered data approximation on S3 and SO(3)]}  that $r:\mathbb S^{3} \to \mathrm{SO(3)}$ is a local isometry, i.e., it respects the Riemannian structures, implying
the relations 
\[
  \begin{aligned}
    d_{\mathbb S^{3}}(x_{1},x_{2}) & = \frac{1}{2} d_{\mathrm{SO(3)}}(r(x_{1}),r(x_{2})), \qquad x_{1},x_{2} \in \mathbb S^{3}, \quad d_{\mathbb S^{3}}(x_{1},x_{2}) \le \frac{\pi}{2},\\
%   \sigma_{\mathbb S^{3}}(r^{-1}(B)) & = \sigma_{\mathrm{SO(3)}}(B), \qquad B \subset \mathrm{SO(3)}. 
    \frac 12 r_{*} \sigma_{\mathbb S^{3}}(B) & =  \sigma_{\mathrm{SO(3)}}(B),\qquad B \subset \mathrm{SO(3)}, \quad \mathrm{\diam}(B) = \sup_{x,y \in B} d_{\mathrm{SO(3)}}(x,y) \le \frac{\pi}{4},\\
%    \int_{\mathrm{SO(3)}} f(x) \mathrm d (r_{*} \sigma_{\mathbb S^{3}})(x) & = \int_{\mathbb S^{3}} f \circ r(x) \mathrm d \sigma_{\mathbb S^{3}}(x)
  \end{aligned}
\]
and maps the even polynomials of degree $2N$ on $\mathbb S^{3}$ to polynomials of degree $N$ on $\mathrm{SO(3)}$, and vice versa, i.e.,
\[
p \in \mathrm{Pol}_{2N}(\mathbb S^{3})\;: \quad p(x) = p(-x),\; x \in \mathbb S^{3}  \qquad \Leftrightarrow \qquad p \circ r^{-1} \in \mathrm{Pol}_{N}(\mathrm{SO(3)}).
\]
Now, let $\tilde \gamma_{N}:[0,1]\to \mathbb S^{3}$ and $\tilde \omega_{N}$ be given as in {[\color{blue} Section $\mathbb S^{d}$]}  for $d=3$ ,i.e., $(\tilde \gamma_{N})_{*} \mathrm d \tilde \omega_{N}$ is an $N$-cubature of speed $\sim N^{2}$ for the sphere $\mathbb S^{3}$.

Then for the push-forward of curve
\[
  \gamma_{N}:[0,1] \to \mathrm{SO(3)},\qquad \gamma_{N}(t) := r \circ  \tilde \gamma_{N}(t),
\]
for the measure $\omega_{N} = \frac12 \tilde \omega_{N}$ we obtain with $p :=\tilde p \circ r^{-1}$, $\tilde p \in \mathrm{Pol}_{2N}(\mathbb S^{3})$, $\tilde p(x)=\tilde p(-x)$, that
\[
  \begin{aligned}
  \int_{\mathrm{SO(3)}} p(x) \mathrm d \sigma_{\mathrm{SO(3)}}(x) 
& =  \int_{\mathrm{SO(3)}} \tilde p \circ r^{-1}(x) \mathrm d \sigma_{\mathrm{SO(3)}}(x) \\
&=  \frac12 \int_{\mathbb S^{3}} \tilde p (x) \circ r^{-1} \circ r (x) \mathrm d \sigma_{\mathbb S^{3}}(x) \\
&=  \frac12 \int_{\mathbb S^{3}} \big(\tilde p (x)+ \tilde p(-x)\big) \mathrm d \sigma_{\mathbb S^{3}}(x) \\
& = \int_{0}^{1} \tilde p \circ \tilde \gamma_{N} (t) \mathrm d \tilde \omega_{N}(t)\\
& = \frac12 \int_{0}^{1} \tilde p \circ r^{-1} \circ r \circ \tilde \gamma_{N} (t) \mathrm d \tilde \omega_{N}(t)\\
& = \int_{0}^{1} p \circ \gamma_{N} (t) \mathrm d \omega_{N}(t).
\end{aligned}
\]
Hence,the property \eqref{eq:quad_curve_Sd} is fulfilled, where the speed of $\gamma_{N}$ is $\sim N^{2}$.

\section{Discrepancy Kernel on $\mathbb T^{2}$}
Emmbeding the the torus $\mathbb T^{2}$ in $\mathbb R^{4}$ by
\[
  x(\alpha_{1},\alpha_{2}) = (\sin(\alpha_{1}),\cos(\alpha_{1}),\sin(\alpha_{2}),\cos(\alpha_{2})), \qquad \alpha_{1},\alpha_{2} \in [0,2\pi), 
\]
the kernel
\[
  K_{p}(x,y) = 2^{-p} \|x-y\|_{2}^{p} = (1-\tfrac 12\cos(\alpha_{1}-\beta_{1})-\tfrac 12\cos(\alpha_{2}-\beta_{2}))^{\frac{p}{2}} 
\]
admits for $p>-2$ the Fourier expansion
\[
  K_{p}(x,y) =\sum_{n \in \mathbb Z^{2}} a_{|n_{1}|,|n_{2}|}(p) \mathrm{e}^{i \mathrm{i} \langle n, \alpha-\beta\rangle}.
\]
\begin{thm}\label{thm:G24}
For $p>-2$, and $n_{1},n_{2} \ge 0$ it holds
\begin{equation}\label{eq:a alpha}
%  a_{n_{1},n_{2}}(p) = \frac{ 4^{-(n_{1}+n_{2})}\Gamma(n_{1}+n_{2}-\frac p2)}{\Gamma(1+n_{1})\Gamma(1+n_{2})\Gamma(-\frac p2)}
  a_{n_{1},n_{2}}(p) =  4^{-(n_{1}+n_{2})} {{-\frac{p}{2}-1}\choose{n_{1}, n_{2}}} \;\pFq{4}{3}{\tfrac{n_{1}+n_{2}+1}{2},\tfrac{n_{1}+n_{2}+2}{2},\tfrac{n_{1}+n_{2}}{2}-\tfrac{p}{4},\tfrac{n_{1}+n_{2}+1}{2}-\tfrac{p}{4}}{n_{1}+n_{2}+1,n_{1}+1,n_{2}+1}{1}.
\end{equation}
In particular, if $p\not\in 2\N$, then 
\begin{equation} \label{eq:dec-G24}
|a_{|n_{1},|n_{2}|}(p)|= \frac{1}{\pi} \bigg|\frac{\Gamma(\frac{p}{2}+1)}{\Gamma(-\frac{p}{2})}\bigg|\|n\|^{-(p+2)} (1+o(1)),
\qquad n \in \mathbb Z^{2},
\end{equation}
and the series terminates if $p\in 2\N$. 
\end{thm}
\begin{proof}
Similar to the proof for the Grassmannian case.
\end{proof}

For $d=1$ we have for
\[
  K_{p}(x,y) = 2^{-\frac p2}\|x-y\|_{2}^{p} = (1-\cos(\alpha-\beta))^{\frac p2} = \sum_{n \in \mathbb Z} a_{|n|}(p) \mathrm{e}^{\mathrm i n (\alpha-\beta)}, 
\]
that the coefficients satisfy
\[
 % a_{n}(p) =  \frac{2^{-n} \Gamma(n-\frac{p}{2})}{\Gamma(n+1)\Gamma(-\frac{p}{2})}
 a_{n}(p) =  2^{-n} {{-\frac{p}{2}-1}\choose{n}}
  \;\pFq{2}{1}{\tfrac{n}{2}-\tfrac{p}{4},\tfrac{n+1}{2}-\tfrac{p}{4}}{n+1}{1}.
\]
\bibliographystyle{abbrv}
\bibliography{biblio_ehler2}
\end{document}
