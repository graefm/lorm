%\documentclass[12pt,a4paper]{amsart}
%!TEX TS-options = --shell-escape
\pdfobjcompresslevel=0% fix PDF inclusions
\documentclass[draft,
a4paper,11pt,DIV=11,% decrease borders by 2 (std 10 for 11pt)
abstract=on% return to old style centered abstract
]{scrartcl}
%
\usepackage{algorithm, booktabs}
\usepackage{mathtools}
\mathtoolsset{showonlyrefs}
\usepackage{amsfonts}
\usepackage{amsmath}
\usepackage{amssymb}
\usepackage{faktor}
\usepackage{amscd}
\usepackage{array}
%\usepackage{refcheck}\norefnames
\usepackage{amsthm}
\usepackage{subfigure}
\usepackage{mathdots}
\usepackage{mathrsfs}
\usepackage[noend]{algpseudocode}
\usepackage{graphicx}
\usepackage{color}
\usepackage{pgf}
\usepackage{scrpage2}
\usepackage{multirow}
\usepackage{listings}
\lstset{language=matlab,showstringspaces=false,basicstyle={\ttfamily}}
\usepackage{tikz}
\usepackage{graphicx}
\usepackage[left=2.5cm, right=2.5cm,top=2.5cm,bottom=2.5cm]{geometry}
\usepackage{scalerel}

\DeclareMathOperator{\Lcurve}{L-curv}
\newcommand{\weakly}{\rightharpoonup}
\newcommand{\N}{\ensuremath{\mathbb{N}}}
\newcommand{\T}{\ensuremath{\mathbb{T}}}
\renewcommand{\S}{\ensuremath{\mathbb{S}}}
\newcommand{\B}{\ensuremath{\mathbb{B}}}
\newcommand{\NZ}{\ensuremath{\mathbb{N}_{0}}}
\newcommand{\Z}{\ensuremath{\mathbb{Z}}}
\newcommand{\Q}{\ensuremath{\mathbb{Q}}}
\newcommand{\R}{\ensuremath{\mathbb{R}}}
\newcommand{\C}{\ensuremath{\mathbb{C}}}
\newcommand{\nchoosek}[2]{\left(\begin{array}{c}#1\\#2\end{array}\right)}
\newcommand{\ii}{\textnormal{i}}
\newcommand{\X}{\mathbb{X}}
\newcommand{\Y}{\mathbb{Y}}
\newcommand{\e}{\textnormal{e}}
\newcommand{\supp}{\textnormal{supp}}
\newcommand{\eip}[1]{\textnormal{e}^{2\pi\ii{#1}}}
\newcommand{\eim}[1]{\textnormal{e}^{-2\pi\ii{#1}}}
\newcommand{\norm}[1]{\left\Vert #1\right\Vert}
\def\invisible#1{\textcolor{white}{#1}}
\newcommand{\ceil}[1]{\encl{\lceil}{#1}{\rceil}}
\newcommand{\floor}[1]{\encl{\lfloor}{#1}{\rfloor}}
\newcommand{\zb}[1]{\ensuremath{\boldsymbol{#1}}}
\newcommand{\pmat}[1]{\begin{pmatrix} #1 \end{pmatrix}}
\newcommand{\set}[1]{\left\{ #1 \right\}}
\newcommand{\abs}[1]{\left| #1 \right |}
\newcommand{\dx}{\,\mathrm{d}}
\newcommand{\tT}{\mathrm{T}}
\newcommand{\uT}{\scriptscriptstyle{T}}
\newcommand{\uuT}{{\scaleto{T}{3pt}}}
\newcommand{\TSP}{\text{TSP}}

\def\3{\ss}
\def\ep{\varepsilon}
\def\la{\lambda}
\def\La{{\mathbf\lambda}}
\def\Ga{\Gamma}
\def\fl#1{\lfloor#1\rfloor}
\def\cl#1{\lceil#1\rceil}

\newmuskip\pFqmuskip

\newcommand*\pFq[6][8]{
  \begingroup % only local assignments
  \pFqmuskip=#1mu\relax
  % make the comma math active
 % \mathcode`\,=\string"8000
  % and define it to be \pFqcomma
  \begingroup\lccode`\~=`\,
  \lowercase{\endgroup\let~}\pFqcomma
  % typeset the formula
  {}_{#2}F_{#3}{\left(\genfrac..{0pt}{}{#4}{#5};#6\right)}%
  \endgroup
}
\newcommand*\pRegFq[6][8]{
  \begingroup % only local assignments
  \pFqmuskip=#1mu\relax
  % make the comma math active
 % \mathcode`\,=\string"8000
  % and define it to be \pFqcomma
  \begingroup\lccode`\~=`\,
  \lowercase{\endgroup\let~}\pFqcomma
  % typeset the formula
  {}_{#2}\tilde{F}_{#3}{\left(\genfrac..{0pt}{}{#4}{#5};#6\right)}%
  \endgroup
}


\newcommand{\pFqcomma}{\mskip\pFqmuskip}


\newcommand*{\bigtimes}{\mathop{\raisebox{-.5ex}{\hbox{\Large{$\times$}}}}}

\DeclareMathOperator{\Lip}{Lip}
\DeclareMathOperator*{\AC}{AC^p}
\DeclareMathOperator*{\Speed}{Speed}
\DeclareMathOperator*{\Length}{Length}
\DeclareMathOperator*{\argmax}{arg\,max}
\DeclareMathOperator*{\diag}{diag}
\DeclareMathOperator*{\trace}{trace}
\DeclareMathOperator*{\diam}{diam}
\DeclareMathOperator*{\spann}{span}
\DeclareMathOperator{\dist}{dist}
\DeclareMathOperator{\F}{F}
\DeclareMathOperator*{\maxdeg}{maxdeg}
\DeclareMathOperator*{\sep}{sep}
\DeclareMathOperator*{\lt}{lt}
\DeclareMathOperator*{\Var}{Var}
\DeclareMathOperator*{\Beta}{Beta}
\DeclareMathOperator{\G}{\mathcal{G}}
\DeclareMathOperator{\geo}{geo}
\DeclareMathOperator{\ch}{ch}
\DeclareMathOperator*{\per}{per}
\DeclareMathOperator*{\id}{id}
\DeclareMathOperator*{\vol}{vol}
\DeclareMathOperator{\Pol}{Pol}
\DeclareMathOperator*{\Real}{Re}
\DeclareMathOperator*{\Imag}{Im}
\DeclareMathOperator*{\disc}{disc}
\DeclareMathOperator{\OOO}{O}
\DeclareMathOperator*{\SO}{SO}
\DeclareMathOperator*{\argmin}{argmin}
\DeclareMathOperator*{\Mapprox}{\mathcal{P}_{\text{res}}(\mathcal M)}
\DeclareMathOperator{\Space}{\mathcal{M}}


\newtheorem{thm}{Theorem}[section]
\newtheorem{lemma}[thm]{Lemma}
\newtheorem{remark}[thm]{Remark}
\newtheorem{definition}[thm]{Definition}
\newtheorem{example}[thm]{Example}
\newtheorem{corollary}[thm]{Corollary}
\newtheorem{proposition}[thm]{Proposition}

%---------------------------------------------------------------------------
\begin{document}
\title{Curve Based Approximation of Measures on Manifolds by Discrepancy Minimization}

\author{
Martin Ehler\footnotemark[1],
\and
Manuel Gr\"af\footnotemark[2],
\and
Sebastian Neumayer\footnotemark[3]
\and
and Gabriele Steidl\footnotemark[3]
}
\maketitle

\date{\today}


\footnotetext[3]{Department of Mathematics,
	TU Kaiserslautern,
	Paul-Ehrlich-Str.~31, D-67663 Kaiserslautern, Germany,
	\{neumayer,steidl\}@mathematik.uni-kl.de
	}
\footnotetext[1]{University of Vienna, Department of Mathematics, Vienna, Austria,
	\{martin.ehler\}@univie.ac.at
	}
\footnotetext[2]{Austrian Academy of Sciences, Acoustics Research Institute, Vienna, Austria,
	\{mgraef\}@kfs.oeaw.ac.at
	}

\begin{abstract}
	\noindent\small
	

\end{abstract}

\section{Main problem}

We like to solve
\begin{equation}
  \label{eq:min_PL}
  \inf_{\nu} \mathscr D(\nu,\mu) \qquad \text{subject to} \qquad \nu \in \mathcal P_{L}^{\Lcurve}(\mathbb X).
\end{equation}
%respectively the relaxed optimization problem
%\begin{equation}
%  \label{eq:min_PL_relaxed}
%  \min_{\gamma \in A_{L}} \mathscr D^{2}(\gamma_{*}(\mathrm dt),\mu) + \alpha \int_{0}^{1} (\|\dot \gamma(t)\|-L)_{+}^{2} \mathrm dt.
%\end{equation}

\section{Numerical Approach}

For fixed $L>0$ and probability measure $\mu$ on $\mathbb X$ we aim to solve the minimization problem
\begin{equation}
  \label{eq:constant_speed_min}
  \min_{x_{1},\dots,x_{N}} \mathscr{D}^{2} \Big(\mu, \frac{1}{N} \sum_{k=1}^{N} \delta_{x_{k}}\Big) \qquad \mathrm{s.t.} \qquad    N \mathrm{dist}(x_{k},x_{k-1}) \le L, \quad k=1,\dots,N.
\end{equation}
Letting $N$ tend to infinity we eventually approach a closed curve $\gamma$, $\gamma(\tfrac kN) \approx x_{k}$ of speed at most $L$ such that the push-forward $\nu = \gamma_{*}(\mathrm dt)$, approaches $\mu$.  

We shall solve the optimization problem approximately by applying the nonlinear conjugate gradient method, cf. {\color{blue} Section ...}, to the following unconstrained formulation
\begin{equation}
  \label{eq:constant_speed_min_relaxed}
  \min_{x_{1},\dots,x_{N}} \mathscr{D}^{2} \Big(\mu, \frac{1}{N} \sum_{k=1}^{N} \delta_{x_{k}}\Big) +   \frac{\alpha}{N} \sum_{k=1}^{N} \big(N \mathrm{dist}(x_{k-1},x_{k}) - L \big)_{+}^{2}
\end{equation}
where $\alpha > 0$ is a penalty parameter. Note, for $\alpha \to \infty$ there is a subsequence of minimizers of \eqref{eq:constant_speed_min_relaxed} which will converge to minimizers of \eqref{eq:constant_speed_min}.

Recall that for given kernels $K:\mathbb X\times \mathbb X \to \mathbb R$ the squared discrepancy can be evaluated by
\[
  \mathscr D^{2}\Big(\mu, \frac 1N \sum_{k=1}^{N} \delta_{x_{k}}\Big) = E_{\mu}(x_{1},\dots,x_{N})
\]
where
\begin{equation}
  \label{eq:E_mu}
  E_{\mu}(x_{1},\dots,x_{N}) :=
    \frac{1}{N^{2}}\sum_{i,j=1}^{N} K(x_{i},x_{j}) - 2\sum_{i=1}^{N} \int_{\mathbb X} K(x_{i},x) \mathrm d\mu(x) + \int_{\mathbb X}\int_{\mathbb X} K(x,y) \mathrm d\mu(x) \mathrm d\mu(y).
  \end{equation}
Using the eigenfunction expansion
\[
 K(x,y) = \sum_{l=0}^{\infty} \lambda_{l} \varphi_{l}(x) \overline{\varphi_{l}(y)} 
\]
we have the alternative evaluation formula
\begin{equation}
  \label{eq:E_mu_Fourier}
  E_{\mu}(x_{1},\dots,x_{N}) = \sum_{l=0}^{\infty} \lambda_{l} \Big| \hat\mu_{l}
  - \frac 1N \sum_{i=1}^{N} \varphi_{l}(x_{i}) \Big|^{2}, \quad\quad \mu_{l} := \int_{\mathbb X} \varphi_{l}(x) \mathrm d\mu(x).
\end{equation}
Both formulas have it pros and cons:
\begin{itemize}
\item The formula \eqref{eq:E_mu} allows for exact evaluation only if the expression
\[
  K_{\mu}(x) := \int_{\mathbb X} K(x,y) \mathrm d\mu(y), \qquad  c_{\mu} := \int_{\mathbb X} K_{\mu}(x) \mathrm d\mu(x),
\]
can be written in closed form. In that case the complexity scales quadratically in the number $N$ of points.

\item The formula \eqref{eq:E_mu_Fourier} allows for exact evaluation only if the kernel has a finite expansion. In that case the complexity scales linearly in the number $N$ of points. 
\end{itemize}
Our approach is to approximate the kernel $K$ by a truncation
\[
  K_{t}(x,y) := \sum_{l=0}^{t} \lambda_{l} \varphi_{l}(x) \overline{\varphi_{l}(y)}
\]
with $t \sim N$ and evaluate formula \eqref{eq:E_mu_Fourier} by fast summation methods, so that the overall complexity is almost linearly in $t$ and $N$. 

{\color{blue} To support the reasonablenes of this approach we may compare the minimizers of $K$ with $K_{t}$ for the uniform measure $\mu$.}
\subsection{Choice of Parameters}

In order to find reasonable (local) solutions of \eqref{eq:min_PL} we shall carefully adjust the parameters in problem \eqref{eq:constant_speed_min_relaxed}, namely the penalty parameter $\alpha$, the number of points $N$ and the polynomial degree $t$.

\textbf{Number of Points.}
From the asymptotic of the path lengths of the traveling sales man problem (TSP) {\color{blue}(see aslo asymptotic for atomic/empirical measures)}
we know that we should choose
\[
  N \gtrsim \mathcal L(\gamma)^{\frac{d}{d-1}},
\]
where
\[
  \mathcal L(\gamma) = \sum_{k=1}^{N} \mathrm{dist}(x_{k-1},x_{k}) \le L
\]
is the length of the resulting curve $\gamma$ going through the points, i.e.,
\[
 \gamma(t_{k}) := x_{k}, \qquad t_{k} = \sum_{i=1}^{k}  \mathrm{dist}(x_{i-1},x_{i})/ \mathcal L(\gamma).
\]

In Figure~\ref{fig:Nexperiment} {(\color{blue} Example environment..?)} we show some results for the two-dimensional torus. 

\textbf{Polynomial degree.} We believe it is reasonable to take $t \sim N^{\frac1d} \sim L^{\frac{1}{d-1}}$. Note, in that case the computational cost is almost linearly in $N$, for $\mathbb X = \mathbb T^{d}, \mathbb S^{2}, \mathrm{SO(3)}, \mathcal G_{2,4}$. In Figure~\ref{fig:t_experiment} {(\color{blue}Example environment..?)} we show some results for the two-dimensional torus. 

\textbf{Penalty Parameter.}
Intuitively, the minimizers of \eqref{eq:constant_speed_min_relaxed} will treat both terms almost equally, i.e., for $N\to \infty$ both terms are of the same order. Hence, our heuristic is to chose the parameter $\alpha$, such that the objective function and the penalty term are of the same order for minimizers of \eqref{eq:constant_speed_min_relaxed} i.e.,
\begin{equation}
  \label{eq:penalty_behavior}
  \frac{\alpha}{N} \sum_{k=1}^{N} \big(N \mathrm{dist}(x_{k-1},x_{k}) - L \big)_{+}^{2} \sim N^{-\frac{2s}{d}} \sim
  \min_{x_{1},\dots,x_{N}} \mathscr{D}^{2} \Big(\mu, \frac{1}{N} \sum_{k=1}^{N} \delta_{x_{k}}\Big) 
\end{equation}
Assuming that for the length $\mathcal L(\gamma) = \sum_{k=1}^{N}\mathrm{dist}(x_{k-1},x_{k})$ of a minimizer $\gamma$ we have $\mathcal L(\gamma) \sim L \sim N^{\frac{d-1}{d}}$, so that $N \mathrm{dist}(x_{k-1},x_{k}) \approx L$, then the value of the penalty term behaves like
\[
  \frac{\alpha}{N} \sum_{k=1}^{N} \big(N \mathrm{dist}(x_{k-1},x_{k}) - L \big)_{+}^{2} \sim  \alpha L^{2} \sim \alpha N^{\frac{2d-2}{d}}
\]
So that a reasonable choice is
\[
  \alpha \sim L^{-\frac{2s}{d-1}-2} \sim N^{\frac{-2s-2d+2}{d}}.
\]

In Figure~\ref{fig:L_alpha_experiment} {(\color{blue}Example environment..?)} we show some results for the two-dimensional torus. 



\subsection{Basic Observations}
For large $L$ we cannot hope to find a solution/global minimizer to problem \eqref{eq:constant_speed_min_relaxed}, since the number of local minimizers appears to increase exponentially {\color{blue}[ ref???]}.

The intention of this section is to provide a case study of the influence/interplay of certain parameters to (local) solutions of
problem \eqref{eq:constant_speed_min_relaxed}, namely
\begin{itemize}
\item the parameters $L$, $N$, $\alpha$
\item initial point/curve
\item choice of kernel (truncation parameter $t$)
\item choice of measure $\mu$ (atomic, singular, regular, disconnected support,...) 
\end{itemize}

In what follows we apply the conjugate gradient method to problem \eqref{eq:constant_speed_min_relaxed} for the torus $\mathbb T^{2}$.

\textbf{Influence of the number of points.} We like to show the dependence on the parameter $N$. In this experiment we increase for fixed $L$ and $\alpha$ the number $N$ of points by subdividing the line segments.  We let $\mu$ be the Lebesgue measure, $L=4$, $\alpha=0.1\cdot 2^{-\frac52 i}$, $i=1,\dots,5$, and consider the kernel $K_{t}$, $t=32$. We start with a local minimizer for $N=10$. Then, we successively increase $N$ by inserting the mid points of the line segments connecting $x_{i}$ and $x_{i+1}$ and perform a local minimization. The results are depicted in Figure~\ref{fig:Nexperiment}.

Note, the choice $\alpha =0.1 (N/5)^{-\frac 52}$ on the diagonal in  Figure~\ref{fig:Nexperiment} {\color{blue}(see section: choice of alpha)}. This shows that the number of points leads to reasonable approximations, where the main features of the curves are preserved.

\begin{figure}
  \begin{tabular}{ccccc}
$N=10$  &  $N=20$   & $N=40$  & $N=80$  & $N=160$ \\
%  \includegraphics[width=0.18\textwidth]{Images/N10_L4_a1000} & 
%  \includegraphics[width=0.18\textwidth]{Images/N20_L4_a1000} & 
%  \includegraphics[width=0.18\textwidth]{Images/N40_L4_a1000} &
%  \includegraphics[width=0.18\textwidth]{Images/N80_L4_a1000} &
%  \includegraphics[width=0.18\textwidth]{Images/N160_L4_a1000}\\
$\mathcal L(\gamma)\approx 4.0000$ &
$\mathcal L(\gamma)\approx 4.0000$ &
$\mathcal L(\gamma)\approx 4.0000$ &
$\mathcal L(\gamma)\approx 4.0000$ &
$\mathcal L(\gamma)\approx 4.0000$\\ 
  \end{tabular}
  \caption{Local minimizers  of \eqref{eq:constant_speed_min_relaxed} for $\mu$ the Lebesgue measure, $L=4$ with kernel $K_{t}$, $t=64$ and increasing $N$ (column-wise) and decreasing $\alpha$ (row-wise). The first row shows the local minimizers for  $\alpha=1000$ and corresponds to curves of constant speed and length $L=4$. In the following rows local minimizers are depicted for $\alpha = 0.1\cdot 2^{-\frac52 i}$,$i=1,\dots,5$. Note, the increase in the length of the curves for decreasing $\alpha$ or increasing $N$, until stagnation for sufficient small $\alpha$ or big $N$. For all minimizer the distance between consecutive points is around $L(\gamma)/N$.}
  \label{fig:Nexperiment}
\end{figure}

\textbf{Influence of the penalty parameter.} In this experiment we like to give an intuition of the parameter $\alpha$. There we choose $\alpha := c L^{-5}$ and observe that indeed the (local) minimizers have length almost proportional to $L$. In the experiment we chose $N$ sufficiently large, see Figure~\ref{fig:Nexperiment}. We note that the distance of consecutive points do vary more for smaller constants $c$, i.e., for higher ratios $L(\gamma)/L$. In other words the measure $\omega$ along the constant speed curve differ more from the Lebesgue measure. This corresponds to the fact that $\alpha$ needs to tend to infinity in order to solve problem \eqref{eq:constant_speed_min}.
\begin{figure}
  \begin{tabular}{cccc}
    % $L=1$  &
               $L=2$ &  $L=4$  & $L=8$ & $L=16$\\
%  \includegraphics[width=0.23\textwidth]{Images/Ramp_L1_i1} & 
%  \includegraphics[width=0.23\textwidth]{Images/Ramp_L2_i1} & 
%  \includegraphics[width=0.23\textwidth]{Images/Ramp_L4_i1} & 
%  \includegraphics[width=0.23\textwidth]{Images/Ramp_L8_i1} &
%  \includegraphics[width=0.23\textwidth]{Images/Ramp_L16_i1} \\
%$\mathcal L(\gamma)\approx 3.92$ &
$\mathcal L(\gamma)\approx 8.01$ &
$\mathcal L(\gamma)\approx 16.23$ &
$\mathcal L(\gamma)\approx 32.62$ &
$\mathcal L(\gamma)\approx 65.90$ \\
  \end{tabular}
   \caption{Local minimizers  of \eqref{eq:constant_speed_min_relaxed} for $\mu$ the measure given by the density $\frac{2\pi-y}{4\pi^{3}} \mathrm dx \mathrm dy$, with kernel $K_{t}$, $t=128$ and several choices of $L$ and $\alpha$. $L$ is increasing column-wise and row-wise we set $\alpha = c L^{-5}$. The constants $c=0.003,0.14,100$ are chosen such that the length of the minimizers satisfy $L(\gamma) / L \approx 4,2,1$, respectively. The density of the measure $\omega$ corresponds to the thickness of the curves.}
  \label{fig:L_alpha_experiment}
\end{figure}

\textbf{Influence of the degree.} Now, we like to show the effect of the truncation parameter $t$, i.e., the degree of $K_{t}$. From the previous observations and discussions we infer that the choice $N \sim L^{2}$ and $\alpha \sim L^{-5}$ are quite reasonable. In this experiment we shall provide some evidence that the choice $t \sim L$ might also be reasonable.
In Figure~\ref{fig:t_experiment} we depict the local minimizers of \eqref{eq:constant_speed_min_relaxed} for fixed parameters $L=2,4,8,16$, whith $\alpha = 0.2 \cdot L^{-5}$, $N=20 L^{2}$. We can observe that for $t = c L$ the corresponding local minimizers have common features. For instance, if $c=2$ ($t \approx L(\gamma)$) the minimizers have mostly vertical and horizantal line segments, similar to a path along a grid. Furthermore for sufficientt large $c$, say $c\ge 4$, the minimizers appear to provide {\color{blue} mercedes-star junctions (oder wie würdet ihr dazu sagen ;)}. 

\begin{figure}
  \begin{tabular}{ccccc}
$t=8$  &  $t=16$   & $t=32$  & $t=64$  & $t=128$ \\
%  \includegraphics[width=0.18\textwidth]{Images/T8_L2} & 
%  \includegraphics[width=0.18\textwidth]{Images/T16_L2} & 
%  \includegraphics[width=0.18\textwidth]{Images/T32_L2} & 
%  \includegraphics[width=0.18\textwidth]{Images/T64_L2} & 
%  \includegraphics[width=0.18\textwidth]{Images/T128_L2} \\
%$\mathcal L(\gamma)\approx 4.07$ &
%$\mathcal L(\gamma)\approx 4.07$ &
%$\mathcal L(\gamma)\approx 4.06$ &
%$\mathcal L(\gamma)\approx 4.06$ &
%$\mathcal L(\gamma)\approx 4.05$\\ 
  \end{tabular}
  \caption{Local minimizers  of \eqref{eq:constant_speed_min_relaxed} for $\mu$ the Lebesgue measure, which show the effect of the degree $t$ of the kernel $K_{t}$. Column-wise we increase $t=8,16,32,64,128$ and row-wise we increase $L=2,4,8,16$ where $\alpha = 0.2 L^{-5}$ and $N=20 L^{2}$. Note, the degree $t$ somehow steers the resolution of the curves. It appears that the spacing of the curves is bounded by $t^{-1}$.  Moreover the choice $t = c L$ leads to similar patterns. For instance, if $c=2$ the patterns are given mostly by horizontal or vertical line segments.}
  \label{fig:t_experiment}
\end{figure}




\subsection{Finding good initial points/curves}

Since the objective function in \eqref{eq:constant_speed_min} is highly non-convex, the main problem is to find good initial points $x_{k} \in \mathbb X$.

Our heuristic is as follows. We start with a curve $\gamma:[0,1]\to \mathbb X$ of small length $L(\gamma)$ and solve the problem \eqref{eq:constant_speed_min_relaxed} for certain $\alpha$ in dependency of the parameter $L$.

\textbf{2d-torus.}
We choose the following dependencies $\alpha = L^{-4}$, $N =11 L^{2}$, $t=3 L$. We start with a circle of radius $r=\frac13$ and solve \eqref{eq:constant_speed_min_relaxed} for $L=3$ using the conjugate gradient method. The obtained minimizer is used as initial point for the next minimization problem, where the parameter $L$ is doubled. The procedure is stopped when the twice the length of the minimizer bigger than the prescribed length.

%\bibliographystyle{abbrv}
%\bibliography{biblio_ehler2}
\end{document}
